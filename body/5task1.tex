\section{Task 1: Identification and Structural Analysis of AI Development Factors}

\subsection{Factor System and Quantification}

To quantitatively describe national AI development capability, a multi-factor indicator system is constructed and organized into a standardized data matrix. Let
\begin{equation}
X = [x_{ij}] \in \mathbb{R}^{n \times p}
\end{equation}
denote the standardized indicator matrix, where $x_{ij}$ represents the normalized value of factor $j$ for country $i$, with $n=10$ countries and $p=24$ indicators.

To eliminate scale effects and ensure cross-country comparability, min--max normalization is applied:
\begin{equation}
x_{ij} = \frac{x_{ij}^{\text{raw}}-\min(x_j)}{\max(x_j)-\min(x_j)} \in [0,1].
\end{equation}
The standardized matrix $X$ is directly used as the input for all subsequent structural analyses.

\subsection{Correlation Structure among AI Development Factors}

Correlation analysis is a fundamental tool for exploring linear relationships among multiple factors and is widely used in exploratory structural analysis. In this task, Pearson correlation coefficients are employed to examine the statistical dependence between AI development factors.

The correlation coefficient between factor $j$ and factor $k$ is defined as
\begin{equation}
r_{jk}=\frac{\sum_{i=1}^{n}(x_{ij}-\bar{x}*j)(x*{ik}-\bar{x}*k)}
{\sqrt{\sum*{i=1}^{n}(x_{ij}-\bar{x}*j)^2\sum*{i=1}^{n}(x_{ik}-\bar{x}*k)^2}},
\end{equation}
forming the correlation matrix
\begin{equation}
R=[r*{jk}] \in \mathbb{R}^{p \times p}.
\end{equation}

To highlight the most significant relationships, a strong-correlation edge set is defined as
\begin{equation}
\mathcal{E}={(j,k)\mid |r_{jk}|>\tau,\ j<k},
\end{equation}
where $\tau$ is a predefined threshold.

\textbf{Results and Interpretation.} \
The Pearson correlation analysis shows that AI development factors exhibit a highly interconnected structure. A large number of indicator pairs display strong positive correlations, indicating substantial co-movement among factors across countries.

Core indicators related to talent, market scale, policy intensity, investment, and infrastructure form dense correlation patterns. This overall correlation structure suggests that AI development capability is characterized by strong system-level linkage rather than independent factor variation.

%图片1:AI发展因素相关性热力图
%数据:correlation_matrix.csv
\begin{figure}[htbp]
  \centering
  \includegraphics[width=0.8\textwidth]{figure/task1/fig1_en_Correlation Heatmap of AI Development Factors.pdf}
  \caption{Correlation Heatmap of AI Development Factors}
  \label{fig:Correlation Heatmap of AI Development Factors}
\end{figure}

\subsection{Structural Grouping via Hierarchical Clustering}

While correlation analysis reveals pairwise relationships, clustering methods are commonly used to uncover higher-level structural organization among factors. To this end, correlation-based distances are introduced to reflect similarity in factor behavior.

The distance between factor $j$ and factor $k$ is defined as
\begin{equation}
d_{jk}=1-|r_{jk}|,
\end{equation}
yielding the distance matrix $D=[d_{jk}]$.

Using the average linkage criterion, the distance between two clusters $C_a$ and $C_b$ is computed as
\begin{equation}
D(C_a,C_b)=\frac{1}{|C_a||C_b|}\sum_{j\in C_a}\sum_{k\in C_b} d_{jk}.
\end{equation}

Hierarchical clustering is then applied to identify data-driven groupings within the AI development factor system.

\textbf{Results and Interpretation.} \
The hierarchical clustering dendrogram reveals that AI development factors are grouped into several coherent clusters. Indicators within each cluster exhibit similar statistical patterns, indicating consistent structural grouping across countries.

Clusters broadly correspond to scale and investment factors, computing and infrastructure factors, and talent-related factors, providing a data-driven partition of the AI development indicator system.

%图片2:AI发展因素层次聚类树状图
%数据:cluster_linkage.csv(correlation_matrix.csv)
\begin{figure}[htbp]
  \centering
  \includegraphics[width=0.8\textwidth]{figure/task1/fig2_en_Hierarchical Clustering Dendrogram of AI Development Factors.pdf}
  \caption{Hierarchical Clustering Dendrogram of AI Development Factors}
  \label{fig:Hierarchical Clustering Dendrogram of AI Development Factors}
\end{figure}

\subsection{Principal Component Analysis of Factor Structure}

Given the high dimensionality and potential redundancy among factors, principal component analysis (PCA) is employed to extract a reduced set of comprehensive components. PCA is widely used for dimensionality reduction while preserving the dominant structural information in multivariate data.

The centered data matrix is given by
\begin{equation}
\tilde{X}=X-\mathbf{1}\bar{X}^T,
\end{equation}
and the covariance matrix is
\begin{equation}
C=\frac{1}{n-1}\tilde{X}^T\tilde{X}.
\end{equation}

Eigen-decomposition of $C$ yields
\begin{equation}
C=V\Lambda V^T,
\end{equation}
where $\Lambda=\mathrm{diag}(\lambda_1,\ldots,\lambda_p)$. The number of retained components $m$ is determined by the cumulative variance criterion
\begin{equation}
\sum_{k=1}^{m}\frac{\lambda_k}{\sum_{j=1}^{p}\lambda_j}\ge \eta,
\end{equation}
where $\eta$ denotes the predefined threshold.

\textbf{Results and Interpretation.} \
The PCA variance contribution results show that the first principal component explains 52.1\% of the total variance. The first four principal components together account for approximately 91.4\% of the cumulative variance.

According to the predefined threshold, four principal components are retained to represent the dominant structure of the original indicator set.

%图片3:PCA 方差贡献率图(柱状图 + 累积折线)
%数据:pca_variance.csv
\begin{figure}[htbp]
  \centering
  \includegraphics[width=0.8\textwidth]{figure/task1/fig3_en_Variance Explained Plot for Principal Components.pdf}
  \caption{Variance Explained Plot for Principal Components}
  \label{fig:Variance Explained Plot for Principal Components}
\end{figure}

\subsection{Relative Importance of AI Development Factors}

To identify key driving factors within the system, the relative importance of each factor is quantified based on PCA results. This approach integrates factor loadings and component contributions to reflect the overall influence of each indicator.

The importance of factor $j$ is defined as
\begin{equation}
I_j=\sum_{k=1}^{m} v_{jk}^2\cdot\frac{\lambda_k}{\sum_{l=1}^{p}\lambda_l},
\end{equation}
where $v_{jk}$ denotes the loading of factor $j$ on principal component $k$.

\textbf{Results and Interpretation.} \
The factor importance ranking shows that explanatory power is concentrated in a limited number of indicators. The Top--k factors account for a substantial proportion of the overall structural variance.

High-importance indicators consistently load on the leading principal components, indicating that a small subset of factors dominates the reduced representation of AI development capability.

%图片4:要素重要性排序条形图(Top–k)+ PCA 载荷热图(辅助说明)
%数据:factor_importance.csv(pca_loadings.csv)
\begin{figure}[htbp]
  \centering
  \includegraphics[width=0.8\textwidth]{figure/task1/fig4_en_Factor Importance Ranking Bar Chart.pdf}
  \caption{Factor Importance Ranking Bar Chart}
  \label{fig:Factor Importance Ranking Bar Chart}
\end{figure}

\subsection{Interaction Patterns among AI Development Factors}

Based on the strong correlation structure, clustering results, and factor importance rankings, interaction patterns among AI development factors are summarized. Rather than establishing strict causal relationships, this analysis focuses on statistically supported promotion and constraint patterns within the factor system.

\textbf{Results and Interpretation.} \
The strong-correlation network exhibits a centralized structure, with a small number of indicators forming dense hubs connected to multiple other factors.

Combined with clustering results, the network suggests that AI development factors interact through several tightly coupled structural modules rather than isolated pairwise relationships.

%图片5:强相关网络图 / 弦图
%数据:strong_correlations.csv + cluster_linkage.csv
\begin{figure}[htbp]
  \centering
  \includegraphics[width=0.8\textwidth]{figure/task1/fig5_en_Strong-Correlation Network or Chord Diagram.pdf}
  \caption{Strong-Correlation Network or Chord Diagram}
  \label{fig:Strong-Correlation Network or Chord Diagram}
\end{figure}

\subsection{Task 1 Summary}

Task~1 presents a unified structural view of national AI development capability by integrating correlation analysis, hierarchical clustering, principal component analysis, and interaction network analysis. Rather than interpreting individual results in isolation, these methods jointly characterize the internal organization of the AI development system.

Using a standardized indicator framework, Task~1 identifies strong system-level linkages among AI development factors, showing that capability differences arise from coordinated patterns across talent, investment, policy, and infrastructure dimensions. The results indicate that these factors form coherent structural groups and exhibit substantial interdependence.

Dimensionality reduction further demonstrates that the high-dimensional factor space can be effectively represented by a small number of comprehensive components, with explanatory power concentrated in a limited subset of key indicators. Interaction analysis confirms that these dominant factors occupy central positions within the system structure.

Overall, Task~1 clarifies both the composition and structural characteristics of AI development capability, providing a compact and consistent analytical foundation for the subsequent tasks of evaluation, forecasting, and investment optimization.
