% =========================================================
% 7 Task 4: Optimization of AI Development Investment Strategy
% (Directory-aligned version: only 7.1 / 7.2 appear in ToC;
% all finer parts are embedded as bold mini-headings.)
% =========================================================

\section{Task 4: Optimization of AI Development Investment Strategy}

Tasks~1--3 have established a unified and internally consistent analytical framework
for evaluating national AI development capability.
Specifically, Task~1 constructed a 24-indicator system and identified key structural
correlations among indicators; Task~2 determined objective indicator weights and fixed
the TOPSIS-based evaluation scheme; Task~3 provided both baseline and forecasted
indicator trajectories for all countries up to 2035.

Building upon these results, Task~4 considers a policy-oriented optimization problem.
Assuming that China allocates an additional 1 trillion RMB in special funds starting
from 2026, the objective of this task is to determine an optimal investment allocation
strategy that maximizes China’s comprehensive AI competitiveness in 2035, while
strictly preserving the evaluation standard and comparison environment established
in the previous tasks.

% =========================================================
\subsection{Model Formulation and Constraints}

\textbf{Problem formulation.}
Let $\mathbf{I} = (I_1, I_2, \ldots, I_p)^\top$ denote the investment allocation vector
over the $p=24$ indicators, where $I_j$ represents the investment (in billion RMB)
allocated to indicator $j$.
The optimization objective is defined as
\begin{equation}
\mathbf{I}^* = \arg\max_{\mathbf{I}} \; S_{\mathrm{CN}}\!\left(X^{2035}(\mathbf{I}); \mathbf{w}\right),
\end{equation}
subject to the total budget constraint
\begin{equation}
\sum_{j=1}^{p} I_j = B, \qquad B = 10000.
\end{equation}
Here, $S_{\mathrm{CN}}(\cdot)$ denotes China’s TOPSIS closeness coefficient under the
fixed weight vector $\mathbf{w}$ obtained in Task~2, and
$X^{2035}(\mathbf{I})$ represents the 2035 indicator matrix in which only China’s
indicator values are affected by the investment decision.

\textbf{Symbols and data interfaces.}
To ensure full comparability with the previous tasks, all structural, weighting, and
forecasting information is treated as exogenous input:
\begin{align}
\mathbf{w} &\leftarrow \text{Task~2 (Entropy Weight Method)},\\
\mathbf{x}^{\mathrm{base}}_{\mathrm{CN}} &\leftarrow \text{Task~3 (China baseline, 2026)},\\
X^{\mathrm{scen}}_{2035} &\leftarrow \text{Task~3 (2035 forecast scenario)},\\
\mathcal{E} &\leftarrow \text{Task~1 (strong correlation structure)}.
\end{align}

For each indicator $j$, the following parameters are introduced:
\begin{itemize}
\item $C_j$: unit investment cost required to increase indicator $j$ by one unit;
\item $\gamma_j$: time-lag discount factor reflecting realization speed;
\item $L_j$: upper bound representing saturation or feasible growth limits;
\item $I_{\min}, I_{\max}$: lower and upper bounds on single-indicator investment.
\end{itemize}

\textbf{Investment--indicator response function.}
Considering diminishing marginal returns and heterogeneous realization horizons
across indicator types, the incremental change of indicator $j$ induced by investment
$I_j$ is modeled as
\begin{equation}
\Delta x_j(\mathbf{I}) = \frac{I_j}{C_j}
\left(1 - \frac{x^{\mathrm{base}}_j}{L_j}\right)\gamma_j,
\qquad j = 1, \ldots, p.
\end{equation}

Accordingly, China’s indicator value in 2035 after investment is given by
\begin{equation}
x^{2035}_{\mathrm{CN},j}(\mathbf{I})
= \min\left\{x^{\mathrm{base}}_j + \Delta x_j(\mathbf{I}), \; L_j\right\}.
\end{equation}

In vector form,
\begin{equation}
\Delta \mathbf{x}(\mathbf{I})
= \left(\boldsymbol{\gamma} \oslash \mathbf{C}\right)
\odot \left(\mathbf{1} - \mathbf{x}^{\mathrm{base}} \oslash \mathbf{L}\right)
\odot \mathbf{I},
\end{equation}
where $\odot$ and $\oslash$ denote element-wise multiplication and division, respectively.

\textbf{Upper bounds and time-lag settings.}
Indicator upper bounds are determined according to a relative competitiveness rule:
\begin{equation}
L_j =
\begin{cases}
1.5 \, x^{\mathrm{CN}}_{j,2025},
& x^{\mathrm{CN}}_{j,2025} \ge x^{\mathrm{US}}_{j,2025},\\[6pt]
3.0 \, x^{\mathrm{US}}_{j,2025},
& x^{\mathrm{CN}}_{j,2025} < x^{\mathrm{US}}_{j,2025},
\end{cases}
\qquad
L_j \le 100 \;\; \text{(ratio-type indicators)}.
\end{equation}

Time-lag discount factors are grouped by indicator characteristics:
\begin{equation}
\gamma_j \in \{1.0,\; 0.8,\; 0.6\},
\end{equation}
corresponding respectively to short-term (e.g., infrastructure), medium-term (e.g., R\&D and applications),
and long-term (e.g., talent cultivation) effects.

\textbf{Evaluation matrix construction.}
The 2035 evaluation matrix $X^{2035}(\mathbf{I}) \in \mathbb{R}^{n \times p}$ is constructed as follows:
all non-China rows are fixed at their forecasted values from Task~3, while China’s row is replaced by
$\mathbf{x}^{2035}_{\mathrm{CN}}(\mathbf{I})$.
This design guarantees that the comparison set and evaluation standard remain unchanged.

\textbf{TOPSIS-based objective function.}
Let $X = X^{2035}(\mathbf{I})$. The TOPSIS procedure follows exactly the formulation in Task~2:
\begin{equation}
\tilde{X} = X D^{-1},
\quad
D = \mathrm{diag}\left(\lVert X_{:,1}\rVert_2, \ldots, \lVert X_{:,p}\rVert_2\right),
\end{equation}
\begin{equation}
V = \tilde{X} \, \mathrm{diag}(\mathbf{w}),
\quad
\mathbf{v}^+ = \max_i V_{i,:},
\quad
\mathbf{v}^- = \min_i V_{i,:},
\end{equation}
\begin{equation}
D_i^{\pm} = \lVert V_{i,:} - \mathbf{v}^{\pm} \rVert_2,
\qquad
S_i = \frac{D_i^-}{D_i^+ + D_i^-}.
\end{equation}
The optimization objective is to maximize $S_{\mathrm{CN}}$.

\textbf{Constraints.}
\textbf{(1) Budget and bound constraints:}
\begin{equation}
\sum_{j=1}^{p} I_j = B,
\qquad
I_{\min} \le I_j \le I_{\max}.
\end{equation}

\textbf{(2) Synergy constraints:}
To prevent structural imbalance caused by isolated investment surges,
synergy constraints are imposed based on strong correlations identified in Task~1:
\begin{align}
x_{\text{Large Models}} &\le 200 \, x_{\text{GPU}},\\
x_{\text{Top AI Scholars}} &\le 5.0 \, x_{\text{Researchers}},\\
x_{\text{AI Publications}} &\le 0.24 \, x_{\text{Researchers}},\\
x_{\text{AI Enterprises}} &\le 78 \, x_{\text{AI Market}},\\
x_{\text{AI Datasets}} &\le 0.75 \, x_{\text{Enterprise R\&D}}.
\end{align}

\textbf{Solution method.}
The resulting optimization problem is a nonlinear constrained programming problem.
It is solved using the Sequential Least Squares Programming (SLSQP) algorithm,
with an equal-allocation initialization $I_j = B/p$,
a maximum of 500 iterations, and a convergence tolerance of $10^{-6}$.

This formulation yields a reproducible investment allocation strategy fully consistent
with the structural insights, evaluation methodology, and forecast scenarios established in Tasks~1--3.

% =========================================================
\subsection{Optimal Allocation Results and Insights}

This subsection reports the optimization outputs produced by Task~4 and interprets them under the
fixed evaluation standard of Tasks~1--3. All investment amounts are in \emph{billion RMB}.

\textbf{Overall allocation pattern.}
The optimized plan exhibits a clear \emph{infrastructure--policy--market} priority, with a secondary focus on
enterprise R\&D and high-end talent. The total allocation over the six TAPRIO dimensions is summarized in
Table~\ref{tab:dim_dist}. The distribution is also visualized in Fig.~\ref{fig:dim_donut}.

\begin{table}[htbp]
\centering
\caption{Dimension-level distribution of the 1 trillion RMB special fund.}
\label{tab:dim_dist}
\begin{tabular}{lrr}
\toprule
Dimension & Investment (billion RMB) & Share (\%) \\
\midrule
Infrastructure (I) & 3232.64 & 32.33 \\
Talent (T)         & 1739.28 & 17.39 \\
Policy (P)         & 1739.11 & 17.39 \\
Application (A)    & 1512.40 & 15.12 \\
R\&D (R)           & 1264.80 & 12.65 \\
Output (O)         &  511.84 &  5.12 \\
\bottomrule
\end{tabular}
\end{table}

\begin{figure}[htbp]
\centering
\includegraphics[width=0.62\textwidth]{figure/task4/fig1_dimension_investment_donut.pdf}
\caption{Dimension-level investment distribution (donut chart).}
\label{fig:dim_donut}
\end{figure}

\textbf{Indicator-level investment priorities.}
Table~\ref{tab:top10_invest} lists the top-10 indicators receiving the largest allocations, which jointly account for
approximately two-thirds of the total budget. Fig.~\ref{fig:top10_lollipop} provides a ranked visualization.

\begin{table}[htbp]
\centering
\caption{Top-10 funded indicators under the optimized allocation.}
\label{tab:top10_invest}
\begin{tabular}{rlrr}
\toprule
Rank & Indicator & Investment (billion RMB) & Share (\%) \\
\midrule
1  & GPU cluster scale                & 1500.00 & 15.00 \\
2  & Number of AI policies            & 1474.23 & 14.74 \\
3  & AI market size                   & 1121.23 & 11.21 \\
4  & Enterprise R\&D expenditure      &  927.66 &  9.28 \\
5  & Top AI scholars                  &  820.92 &  8.21 \\
6  & AI researchers                   &  816.90 &  8.17 \\
7  & TOP500 supercomputer count       &  731.54 &  7.32 \\
8  & GitHub AI-related projects       &  308.86 &  3.09 \\
9  & Number of data centers           &  281.74 &  2.82 \\
10 & Internet bandwidth               &  232.86 &  2.33 \\
\bottomrule
\end{tabular}
\end{table}

\begin{figure}[htbp]
\centering
\includegraphics[width=0.78\textwidth]{figure/task4/fig2_top10_investment_lollipop.pdf}
\caption{Top-10 indicator investments (lollipop chart).}
\label{fig:top10_lollipop}
\end{figure}

\textbf{Full 24-indicator allocation.}
For completeness and reproducibility, Table~\ref{tab:all24_invest} reports the full 24-indicator allocation.

\begin{table}[htbp]
\centering
\caption{Full investment allocation across all 24 indicators.}
\label{tab:all24_invest}
\small
\begin{tabular}{rlrr}
\toprule
Rank & Indicator & Investment (billion RMB) & Share (\%) \\
\midrule
1  & GPU cluster scale                & 1500.000 & 15.000 \\
2  & Number of AI policies            & 1474.225 & 14.742 \\
3  & AI market size                   & 1121.225 & 11.212 \\
4  & Enterprise R\&D expenditure      &  927.661 &  9.277 \\
5  & Top AI scholars                  &  820.922 &  8.209 \\
6  & AI researchers                   &  816.904 &  8.169 \\
7  & TOP500 supercomputer count       &  731.544 &  7.315 \\
8  & GitHub AI-related projects       &  308.863 &  3.089 \\
9  & Number of data centers           &  281.743 &  2.817 \\
10 & Internet bandwidth               &  232.859 &  2.329 \\
11 & Number of AI enterprises         &  218.799 &  2.188 \\
12 & International AI investment      &  189.042 &  1.890 \\
13 & AI subsidy amount                &  149.758 &  1.498 \\
14 & Government AI investment         &  148.102 &  1.481 \\
15 & AI computing platforms           &  145.773 &  1.458 \\
16 & Electricity production           &  138.211 &  1.382 \\
17 & AI application penetration       &  122.376 &  1.224 \\
18 & AI social trust                  &  115.123 &  1.151 \\
19 & Number of AI datasets            &  101.499 &  1.015 \\
20 & Number of AI books               &  101.482 &  1.015 \\
21 & Number of AI graduates           &  101.459 &  1.015 \\
22 & 5G coverage rate                 &  101.438 &  1.014 \\
23 & Internet penetration rate        &  101.075 &  1.011 \\
24 & Number of large models           &   50.000 &  0.500 \\
\bottomrule
\end{tabular}
\normalsize
\end{table}

\textbf{Indicator improvements under the response function.}
Using the response function in Eq.~(8)--(10), we compute China’s post-investment indicator levels in 2035.
Table~\ref{tab:before_after} reports the baseline (2026) and the corresponding post-investment (2035) values
generated by the Task~4 pipeline. Fig.~\ref{fig:growth_lollipop} visualizes the growth rates on a log scale.

\begin{table}[htbp]
\centering
\caption{Selected indicator changes from baseline (2026) to post-investment level (2035).}
\label{tab:before_after}
\small
\begin{tabular}{lrrrr}
\toprule
Indicator & Baseline 2026 & Post-invest 2035 & Increment & Growth (\%) \\
\midrule
GitHub AI-related projects & 5094.574 & 99000.000 & 234374.899 & 4600.481 \\
AI graduates              &   70.000 &    97.500 &   1716.970 & 2452.803 \\
GPU cluster scale         &    3.967 &    32.968 &     29.001 &  731.113 \\
Internet bandwidth        &    1.607 &     6.201 &      4.594 &  285.955 \\
Number of AI policies     &   72.933 &   270.580 &    197.647 &  270.996 \\
AI researchers            &  279.342 &   900.000 &    676.024 &  242.006 \\
TOP500 supercomputer cnt  &   83.200 &   205.780 &    122.580 &  147.332 \\
AI market size            &  138.000 &   295.159 &    157.159 &  113.883 \\
AI enterprises            & 5901.133 & 11830.887 &   5929.754 &  100.485 \\
Enterprise R\&D exp.      &  665.333 &  1172.337 &    507.003 &   76.203 \\
\bottomrule
\end{tabular}
\normalsize
\end{table}

\begin{figure}[htbp]
\centering
\includegraphics[width=0.82\textwidth]{figure/task4/fig3_growth_rate_lollipop_log.pdf}
\caption{Indicator growth rates under the optimized investment (log-scale lollipop).}
\label{fig:growth_lollipop}
\end{figure}

Two practical patterns emerge from the computed responses:
\begin{itemize}
\item \textbf{Large, scalable digital-production indicators} (e.g., GitHub projects) exhibit the highest percentage growth,
consistent with a strong ``capacity expansion'' effect once infrastructure and R\&D are funded.
\item \textbf{Saturated ratio-type indicators} (e.g., 5G coverage) show limited or zero increments, reflecting the upper-bound
and diminishing-return mechanisms embedded in $L_j$ and the factor $\left(1-\frac{x^{\mathrm{base}}_j}{L_j}\right)$.
\end{itemize}

\textbf{Impact on 2035 TOPSIS competitiveness.}
Under the fixed TOPSIS evaluation procedure (Task~2) and the 2035 comparison set (Task~3 scenario),
China’s closeness coefficient after optimization is
\begin{equation}
S_{\mathrm{CN}}^{2035,\;\mathrm{post}} = 0.54717.
\end{equation}
For reference, the pipeline reports China’s baseline score as $S_{\mathrm{CN}}^{2026}=0.592821$,
yielding a net change of $-0.045651$ (i.e., $-7.70\%$) when comparing \emph{2026 baseline} with \emph{2035 post-investment}.
This does \emph{not} contradict the optimization goal: the objective maximizes $S_{\mathrm{CN}}$ \emph{within the 2035 scenario},
while the cross-year decline indicates that the global benchmark (other countries’ forecasted 2035 levels) becomes more demanding.
In other words, even with additional funding, China’s relative position in 2035 is evaluated against a
more advanced global frontier.

\textbf{Cross-dimension comparison and efficiency diagnostics.}
To better interpret \emph{why} the optimized plan concentrates on certain dimensions and indicators,
we provide two complementary diagnostics.

\textbf{(1) Unified dimension comparison.}
Fig.~\ref{fig:dim_compare} summarizes China’s dimension-level changes under the unified TAPRIO grouping,
which is consistent with the dimension aggregation used throughout Tasks~1--3.

\begin{figure}[htbp]
\centering
\includegraphics[width=0.82\textwidth]{figure/task4/fig4_unified_dimension_comparison.pdf}
\caption{Unified TAPRIO dimension comparison before vs.\ after investment (2035).}
\label{fig:dim_compare}
\end{figure}

\textbf{(2) Investment efficiency quadrant.}
Fig.~\ref{fig:eff_quadrant} plots indicators in an efficiency space (bubble chart),
supporting a qualitative classification into:
(i) ``high investment--high return'' levers (typically infrastructure and key inputs),
(ii) ``low investment--high return'' quick wins (usually near-unsaturated digital indicators),
and (iii) ``high investment--low return'' saturated or lagged indicators,
which are naturally down-weighted by the optimization under diminishing returns and time-lag discounts.

\begin{figure}[htbp]
\centering
\includegraphics[width=0.82\textwidth]{figure/task4/fig5_bubble_efficiency_quadrant.pdf}
\caption{Efficiency quadrant for indicator investments (bubble chart).}
\label{fig:eff_quadrant}
\end{figure}

\textbf{Actionable recommendations.}
Finally, to translate the optimized vector $\mathbf{I}^*$ into implementable policy directions,
we summarize the plan into three tiers:
\begin{enumerate}
\item \textbf{Strategic capacity foundation (I \& P)}: prioritize GPU clusters, TOP500 capacity, data centers, and
a coherent policy package to avoid bottlenecks in compute and governance.
\item \textbf{Innovation production engine (A \& R)}: expand AI market and enterprise R\&D to convert infrastructure
into scalable industrial output and application diffusion.
\item \textbf{Talent upgrading (T)}: target high-end researchers and top scholars as the ``multiplier'' for the R\&D--application loop,
consistent with the synergy constraints in Task~1.
\end{enumerate}

All results above are generated strictly under the Task~4 response function, constraints, and the fixed
Task~2 TOPSIS evaluation scheme, ensuring full model consistency and reproducibility.
