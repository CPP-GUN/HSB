% =========================
% body/8task4.tex
% =========================
\FloatBarrier
\section{Task 4: Optimization of AI Development Investment Strategy}

\begin{figure}[htbp]
  \centering
  \includegraphics[width=0.95\textwidth]{figure/task4/lct.png}
  \caption{Overall decision and optimization framework of Task~4.}
  \label{fig:task4_framework}
\end{figure}

Fig.~\ref{fig:task4_framework} summarizes the end-to-end optimization pipeline of Task~4.
Starting from the baseline trajectories obtained in Task~3 and a fixed total budget,
an investment--response model with diminishing returns and time-lag effects is constructed.
The resulting nonlinear constrained problem is solved using SLSQP,
and the optimized allocation is evaluated under the same TOPSIS framework as in Task~2,
yielding allocation patterns, competitiveness gains, and policy insights.

Tasks~1--3 establish a consistent pipeline: indicators $\rightarrow$ objective weights $\rightarrow$ TOPSIS scores $\rightarrow$ multi-year scenario.
Task~4 turns this pipeline into a decision problem. Starting in 2026, China allocates an additional \emph{1 trillion RMB}
special fund, and the goal is to maximize China’s \emph{2035} comprehensive AI competitiveness under the \emph{same}
evaluation standard and comparison set.

\subsection{Model Formulation and Constraints}

\textbf{Decision variables and budget.}
Let $\mathbf{I}=(I_1,\dots,I_p)^\top$ be the allocation across $p=24$ indicators.
To match the policy budget and the numerical outputs, we measure investment in \emph{hundred-million RMB} (100 million yuan).
Thus, the total budget is
\begin{equation}
\sum_{j=1}^{p} I_j = B,\qquad B=10000\ \text{(100 million yuan)}.
\label{eq:t4_budget}
\end{equation}

\textbf{Objective (fixed evaluation standard).}
Let $S_{\mathrm{CN}}(\cdot)$ denote China’s TOPSIS closeness coefficient under the fixed weight vector $\mathbf{w}$
obtained in Task~2. The optimization objective is
\begin{equation}
\mathbf{I}^*=\arg\max_{\mathbf{I}}\; S_{\mathrm{CN}}\!\left(X^{2035}(\mathbf{I});\mathbf{w}\right),
\label{eq:t4_obj}
\end{equation}
where $X^{2035}(\mathbf{I})$ is the 2035 evaluation matrix: all non-China rows are fixed at the Task~3 scenario values,
and only China’s row is updated by the investment response.

\textbf{Inputs from previous tasks.}
\begin{align}
\mathbf{w} &\leftarrow \text{Task~2 (EWM weights)},\\
\mathbf{x}^{\mathrm{base}}_{\mathrm{CN}} &\leftarrow \text{Task~3 (China baseline trajectory)},\\
X^{\mathrm{scen}}_{2035} &\leftarrow \text{Task~3 (2035 scenario for all countries)},\\
\mathcal{E} &\leftarrow \text{Task~1 (strong correlation structure)}.
\end{align}

\textbf{Investment--indicator response (diminishing returns \& time lag).}
For indicator $j$, introduce: unit cost $C_j$, time-lag discount $\gamma_j$, and saturation upper bound $L_j$.
The investment-induced increment is modeled by
\begin{equation}
\Delta x_j(\mathbf{I})
=\frac{I_j}{C_j}\left(1-\frac{x^{\mathrm{base}}_j}{L_j}\right)\gamma_j,
\qquad j=1,\dots,p,
\label{eq:t4_response}
\end{equation}
and the post-investment level is truncated by feasibility:
\begin{equation}
x^{2035}_{\mathrm{CN},j}(\mathbf{I})
=\min\left\{x^{\mathrm{base}}_j+\Delta x_j(\mathbf{I}),\; L_j\right\}.
\label{eq:t4_post}
\end{equation}
Upper bounds follow a relative-competitiveness rule:
\begin{equation}
L_j =
\begin{cases}
1.5 \, x^{\mathrm{CN}}_{j,2025},
& x^{\mathrm{CN}}_{j,2025} \ge x^{\mathrm{US}}_{j,2025},\\[6pt]
3.0 \, x^{\mathrm{US}}_{j,2025},
& x^{\mathrm{CN}}_{j,2025} < x^{\mathrm{US}}_{j,2025},
\end{cases}
\qquad
L_j \le 100\ \text{(ratio-type indicators)}.
\end{equation}
Time-lag discounts are grouped as $\gamma_j\in\{1.0,0.8,0.6\}$ for short-/medium-/long-horizon effects.

\textbf{TOPSIS evaluation (same as Task~2).}
Let $X=X^{2035}(\mathbf{I})$. Using vector normalization,
\begin{equation}
\tilde{X}=XD^{-1},\quad
D=\mathrm{diag}\!\left(\|X_{:,1}\|_2,\dots,\|X_{:,p}\|_2\right),
\end{equation}
\begin{equation}
V=\tilde{X}\,\mathrm{diag}(\mathbf{w}),\quad
\mathbf{v}^+=\max_i V_{i,:},\quad
\mathbf{v}^-=\min_i V_{i,:},
\end{equation}
\begin{equation}
D_i^{\pm}=\|V_{i,:}-\mathbf{v}^{\pm}\|_2,
\qquad
S_i=\frac{D_i^-}{D_i^+ + D_i^-}.
\end{equation}
The objective \eqref{eq:t4_obj} maximizes $S_{\mathrm{CN}}$.

\textbf{Constraints.}
\textbf{(1) Budget and bounds:}
\begin{equation}
\sum_{j=1}^{p} I_j=B,\qquad I_{\min}\le I_j\le I_{\max}.
\end{equation}
\textbf{(2) Synergy constraints (from Task~1 strong links).}
To avoid structurally imbalanced growth, we impose ratio-type coupling constraints:
\begin{align}
x_{\text{Large Models}} &\le 200 \, x_{\text{GPU}},\nonumber\\
x_{\text{Top AI Scholars}} &\le 5.0 \, x_{\text{Researchers}},\nonumber\\
x_{\text{AI Publications}} &\le 0.24 \, x_{\text{Researchers}},\nonumber\\
x_{\text{AI Enterprises}} &\le 78 \, x_{\text{AI Market}},\nonumber\\
x_{\text{AI Datasets}} &\le 0.75 \, x_{\text{Enterprise R\&D}}.\label{eq:t4_synergy}
\end{align}

\textbf{Solution method.}
The nonlinear constrained program is solved by SLSQP with equal-allocation initialization $I_j=B/p$,
maximum 500 iterations, and tolerance $10^{-6}$.

\subsection{Optimal Allocation Results and Insights}

All allocations below are in \emph{100 million yuan (RMB)}.

\textbf{Overall allocation pattern.}
The optimized plan prioritizes \emph{infrastructure--policy--market}, with secondary emphasis on enterprise R\&D and high-end talent.
Table~\ref{tab:dim_dist} and Fig.~\ref{fig:dim_donut} summarize the TAPRIO dimension distribution.

\begin{table}[htbp]
\centering
\caption{Dimension-level distribution of the 1 trillion RMB special fund.}
\label{tab:dim_dist}
\begin{tabular}{lrr}
\toprule
Dimension & Investment (100 million yuan) & Share (\%) \\
\midrule
Infrastructure (I) & 3232.64 & 32.33 \\
Talent (T)         & 1739.28 & 17.39 \\
Policy (P)         & 1739.11 & 17.39 \\
Application (A)    & 1512.40 & 15.12 \\
R\&D (R)           & 1264.80 & 12.65 \\
Output (O)         &  511.84 &  5.12 \\
\bottomrule
\end{tabular}
\end{table}

\begin{figure}[htbp]
\centering
\includegraphics[width=0.62\textwidth]{\detokenize{figure/task4/fig1_dimension_investment_donut.pdf}}
\caption{Dimension-level investment distribution (donut chart).}
\label{fig:dim_donut}
\end{figure}

\textbf{Indicator-level priorities.}
Top-10 funded indicators account for the majority of the budget (Table~\ref{tab:top10_invest}),
and Fig.~\ref{fig:top10_lollipop} visualizes the allocation rank.

\begin{table}[htbp]
\centering
\caption{Top-10 funded indicators under the optimized allocation.}
\label{tab:top10_invest}
\begin{tabular}{rlrr}
\toprule
Rank & Indicator & Investment (100 million yuan) & Share (\%) \\
\midrule
1  & GPU cluster scale                & 1500.00 & 15.00 \\
2  & Number of AI policies            & 1474.23 & 14.74 \\
3  & AI market size                   & 1121.23 & 11.21 \\
4  & Enterprise R\&D expenditure      &  927.66 &  9.28 \\
5  & Top AI scholars                  &  820.92 &  8.21 \\
6  & AI researchers                   &  816.90 &  8.17 \\
7  & TOP500 supercomputer count       &  731.54 &  7.32 \\
8  & GitHub AI-related projects       &  308.86 &  3.09 \\
9  & Number of data centers           &  281.74 &  2.82 \\
10 & Internet bandwidth               &  232.86 &  2.33 \\
\bottomrule
\end{tabular}
\end{table}

\begin{figure}[htbp]
\centering
\includegraphics[width=0.78\textwidth]{\detokenize{figure/task4/fig2_top10_investment_lollipop.pdf}}
\caption{Top-10 indicator investments (lollipop chart).}
\label{fig:top10_lollipop}
\end{figure}

\textbf{Full 24-indicator allocation (reproducibility).}
\begin{table}[htbp]
\centering
\caption{Full investment allocation across all 24 indicators.}
\label{tab:all24_invest}
\scriptsize
\renewcommand{\arraystretch}{0.95}
\begin{tabular}{rlrr}
\toprule
Rank & Indicator & Investment (100 million yuan) & Share (\%) \\
\midrule
1  & GPU cluster scale                & 1500.000 & 15.000 \\
2  & Number of AI policies            & 1474.225 & 14.742 \\
3  & AI market size                   & 1121.225 & 11.212 \\
4  & Enterprise R\&D expenditure      &  927.661 &  9.277 \\
5  & Top AI scholars                  &  820.922 &  8.209 \\
6  & AI researchers                   &  816.904 &  8.169 \\
7  & TOP500 supercomputer count       &  731.544 &  7.315 \\
8  & GitHub AI-related projects       &  308.863 &  3.089 \\
9  & Number of data centers           &  281.743 &  2.817 \\
10 & Internet bandwidth               &  232.859 &  2.329 \\
11 & Number of AI enterprises         &  218.799 &  2.188 \\
12 & International AI investment      &  189.042 &  1.890 \\
13 & AI subsidy amount                &  149.758 &  1.498 \\
14 & Government AI investment         &  148.102 &  1.481 \\
15 & AI computing platforms           &  145.773 &  1.458 \\
16 & Electricity production           &  138.211 &  1.382 \\
17 & AI application penetration       &  122.376 &  1.224 \\
18 & AI social trust                  &  115.123 &  1.151 \\
19 & Number of AI datasets            &  101.499 &  1.015 \\
20 & Number of AI books               &  101.482 &  1.015 \\
21 & Number of AI graduates           &  101.459 &  1.015 \\
22 & 5G coverage rate                 &  101.438 &  1.014 \\
23 & Internet penetration rate        &  101.075 &  1.011 \\
24 & Number of large models           &   50.000 &  0.500 \\
\bottomrule
\end{tabular}
\normalsize
\end{table}

\textbf{Indicator improvements under the response function.}
Using Eqs.~\eqref{eq:t4_response}--\eqref{eq:t4_post}, we compute China’s post-investment indicator levels in 2035.
Table~\ref{tab:before_after} reports selected indicators (baseline vs.\ post-investment), and Fig.~\ref{fig:growth_lollipop}
visualizes growth rates (log-scale).

\begin{table}[htbp]
\centering
\caption{Selected indicator changes from baseline (2026) to post-investment level (2035).}
\label{tab:before_after}
\small
\begin{tabular}{lrrrr}
\toprule
Indicator & Baseline 2026 & Post-invest 2035 & Increment & Growth (\%) \\
\midrule
GitHub AI-related projects & 5094.574 & 99000.000 & 93905.426 & 1843.244 \\
AI graduates              &   70.000 &    97.500 &    27.500 &   39.286 \\
GPU cluster scale         &    3.967 &    32.968 &    29.001 &  731.056 \\
Internet bandwidth        &    1.607 &     6.201 &     4.594 &  285.874 \\
Number of AI policies     &   72.933 &   270.580 &   197.647 &  270.998 \\
AI researchers            &  279.342 &   900.000 &   620.658 &  222.186 \\
TOP500 supercomputer cnt  &   83.200 &   205.780 &   122.580 &  147.332 \\
AI market size            &  138.000 &   295.159 &   157.159 &  113.883 \\
AI enterprises            & 5901.133 & 11830.887 &  5929.754 &  100.485 \\
Enterprise R\&D exp.      &  665.333 &  1172.337 &   507.004 &   76.203 \\
\bottomrule
\end{tabular}
\normalsize
\end{table}

\begin{figure}[htbp]
\centering
\includegraphics[width=0.82\textwidth]{\detokenize{figure/task4/fig3_growth_rate_lollipop_log.pdf}}
\caption{Indicator growth rates under the optimized investment (log-scale lollipop).}
\label{fig:growth_lollipop}
\end{figure}

\textbf{Impact on 2035 TOPSIS competitiveness (within-year comparison).}
Under the fixed TOPSIS procedure (Task~2) and the 2035 comparison set (Task~3 scenario),
China’s post-investment closeness coefficient is
\begin{equation}
S_{\mathrm{CN}}^{2035,\mathrm{post}}=0.54717.
\end{equation}
From Task~3 (no additional investment), China’s 2035 baseline score is $C_{\mathrm{CN},2035}=0.507$ (Table~\ref{tab:task3_scores_selected}),
so the optimized plan yields an improvement of approximately $+0.040$ under the same 2035 benchmark environment.

\textbf{Complementary diagnostics and policy translation.}
Fig.~\ref{fig:t4_diag_pair} links the optimized allocation to (i) dimension-level changes and (ii) investment efficiency patterns.

\begin{figure}[htbp]
\centering
\begin{subfigure}[t]{0.49\textwidth}
  \centering
  \includegraphics[width=\linewidth]{\detokenize{figure/task4/fig4_unified_dimension_comparison.pdf}}
  \caption{TAPRIO dimension comparison (2035).}
  \label{fig:dim_compare}
\end{subfigure}\hfill
\begin{subfigure}[t]{0.49\textwidth}
  \centering
  \includegraphics[width=\linewidth]{\detokenize{figure/task4/fig5_bubble_efficiency_quadrant.pdf}}
  \caption{Efficiency quadrant (bubble chart).}
  \label{fig:eff_quadrant}
\end{subfigure}
\caption{Interpretation tools for the optimized investment plan.}
\label{fig:t4_diag_pair}
\end{figure}

\textbf{Actionable recommendations.}
\begin{enumerate}
\item \textbf{Strategic capacity foundation (I \& P):} prioritize GPU clusters, TOP500 capacity, data centers, and a coherent policy package to avoid compute and governance bottlenecks.
\item \textbf{Innovation production engine (A \& R):} expand AI market size and enterprise R\&D to convert funded capacity into scalable applications and industrial output.
\item \textbf{Talent upgrading (T):} target high-end researchers and top scholars as multipliers, consistent with the synergy constraints in \eqref{eq:t4_synergy}.
\end{enumerate}

All results above are generated under the fixed Task~2 evaluation scheme and the Task~3 2035 scenario,
ensuring model consistency and reproducibility.

\FloatBarrier