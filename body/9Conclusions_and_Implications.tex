\section{Conclusions and Implications}

This study develops a \textbf{unified, data-driven} modeling framework for evaluating, comparing, forecasting, and optimizing national artificial intelligence (AI) development capability. Built upon a consistent system of \textbf{24 indicators} and a unified evaluation protocol, the framework integrates factor identification (Problem~1), comprehensive evaluation (Problem~2), temporal forecasting (Problem~3), and resource optimization (Problem~4) into a \textbf{closed-loop analytical pipeline}, ensuring cross-country and cross-period comparability as well as full result traceability.

\subsection*{$\blacktriangleright$ Key Findings by Problem}

\subsubsection*{$\blacklozenge$ Problem~1.}
National AI development is not driven by isolated factors but emerges from a tightly coupled system of infrastructure, human capital, policy environment, and innovation output. Multiple indicator pairs exhibit strong positive Pearson correlations (with several $|r|>0.7$), highlighting the necessity of \textbf{synergistic} investment strategies rather than fragmented interventions.

\subsubsection*{$\blacklozenge$ Problem~2.}
The 2025 evaluation reveals a clear stratification of AI competitiveness: the United States ranks first (TOPSIS score $=0.641$), China second ($0.510$), and India third ($0.210$). The high consistency between TOPSIS and Grey Relational Analysis rankings confirms the \textbf{robustness} of the results, suggesting that the observed hierarchy is primarily driven by long-term structural advantages rather than short-term fluctuations.

\subsubsection*{$\blacklozenge$ Problem~3.}
Extending the analysis to 2026--2035, the forecasting results indicate strong inertia in the global AI competitiveness landscape under \textbf{structural-stability assumptions}. Leading countries (the United States, China, and India) maintain their advantages in the medium term, while only limited rank exchanges occur among mid-tier countries. Backtesting diagnostics report a median forecasting error of $\textbf{MAPE}=0.1035$ (10.35\%), indicating acceptable predictive accuracy.

\subsubsection*{$\blacklozenge$ Problem~4.}
Under the scenario of an additional CNY~1 trillion investment for China starting in 2026, the optimization results recommend prioritizing infrastructure development (32.33\%, approximately CNY~323.3~billion), followed by talent cultivation and policy support (each 17.39\%). This allocation reflects the high marginal returns of \textbf{computational capacity} and foundational capabilities, while underscoring the complementary roles of institutional design and human capital.

\subsection*{$\blacktriangleright$ Methodological Contributions}

Methodologically, the proposed framework integrates \textbf{entropy-based objective weighting (EWM)}, multi-model cross-validation (TOPSIS and Grey Relational Analysis), time-series forecasting (GM(1,1) with backtesting diagnostics), and nonlinear constrained optimization (SLSQP). Robustness checks and error diagnostics support the credibility of the findings, indicating that the observed ranking patterns are more likely to reflect structural characteristics rather than methodological artifacts.

\subsection*{$\blacktriangleright$ Limitations}

Several limitations should be acknowledged. First, the analysis relies primarily on publicly available data, which may be subject to time lags and may not fully capture recent technological breakthroughs or abrupt policy changes. Second, the forecasting framework assumes structural stability and does not explicitly model disruptive shocks. Third, the investment-response mechanism adopts simplifying assumptions (e.g., diminishing marginal returns), whereas real-world input--output dynamics may involve more complex institutional and market feedbacks.

\subsection*{$\blacktriangleright$ Policy Implications}

Based on the findings, the following recommendations are offered:
\begin{enumerate}
    \item Formulate \textbf{long-term} AI development strategies rather than fragmented short-term interventions;
    \item Prioritize investment in \textbf{computational infrastructure}, allocating above-average shares in the short to medium term to alleviate binding constraints;
    \item Coordinate investments in talent, policy support, and R\&D so that financial, human, and institutional capacities reinforce one another;
    \item For latecomer countries, design catch-up strategies combining large-scale investment, institutional innovation, and international cooperation to mitigate path-dependence effects.
\end{enumerate}

\subsection*{$\blacktriangleright$ Overall Conclusion}

Overall, this study provides a \textbf{transparent and reproducible} framework for national AI capability assessment and optimization, offering quantitative and evidence-based support for policy formulation. Despite limitations in data availability and modeling assumptions, the framework yields meaningful insights into the structural drivers and long-term evolution of national AI development and can be readily extended in future research.
