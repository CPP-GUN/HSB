\section{Task 2: AI Development Capability Evaluation and 2025 Ranking}

\subsection{Model Overview}

Based on the 24 indicators identified in Task 1, let the normalized indicator matrix be
\begin{equation}
X'=\left(x'_{ij}\right)_{n\times p},\quad n=10,\;p=24,
\end{equation}
where $x'_{ij}$ denotes the normalized value of indicator $j$ for country $i$.
All indicators have been unified as benefit-type (larger values indicate stronger AI development capability)
and normalized prior to this task.

The objective of Task 2 is to construct an \textit{objective and reproducible} evaluation model
to quantify national AI development capability and determine the 2025 competitiveness ranking
of ten countries.

To reduce subjective bias and enhance robustness, an integrated evaluation framework combining
the Entropy Weight Method (EWM), TOPSIS, and Grey Relational Analysis (GRA) is adopted.

\subsection{Entropy-Based Weighting}

The entropy weight method is derived from information theory and assigns indicator weights
according to their dispersion across countries. Indicators with higher variability contain
more effective information for discrimination and thus receive larger weights.

Define
\begin{equation}
p_{ij}=\frac{x'_{ij}}{\sum_{i=1}^{n}x'_{ij}},\qquad 0\ln 0 := 0,
\end{equation}
where $p_{ij}$ represents the proportion of indicator $j$ contributed by country $i$.
The information entropy of indicator $j$ is computed as
\begin{equation}
e_j=-k\sum_{i=1}^{n}p_{ij}\ln p_{ij}, \quad k=\frac{1}{\ln n}.
\end{equation}
The corresponding entropy weight is given by
\begin{equation}
w_j=\frac{1-e_j}{\sum_{j=1}^{p}(1-e_j)}.
\end{equation}
If $\sum_{i=1}^{n}x'_{ij}=0$ for some indicator $j$, its weight is set to zero and the remaining
weights are renormalized.

\paragraph{Result interpretation.}
The entropy-based weighting results indicate a clear concentration of information contribution among a limited subset of indicators.
Specifically, frontier capability and high-level output indicators receive substantially larger weights than basic penetration metrics.
The largest weights are assigned to \textit{Large Models} ($w=0.0946$), \textit{International AI Investment} ($w=0.0859$),
\textit{Top AI Scholars} ($w=0.0699$), and \textit{AI Market Size} ($w=0.0698$), implying that cross-country differences are most strongly
distinguished by cutting-edge innovation capacity, global resource linkage, and market maturity.
In contrast, \textit{Internet Penetration} ($w=0.0063$), \textit{AI Penetration} ($w=0.0084$), and \textit{5G Coverage} ($w=0.0087$)
carry relatively small weights, suggesting that foundational digital access has partially converged among major economies and is less
decisive in differentiating AI competitiveness at the current stage.

Overall, the weight structure is consistent with the structural findings in Task~1: a small set of dominant factors explain most of the
cross-country variance, while low-weight indicators provide complementary but limited marginal discrimination.

\begin{figure}[htbp]
  \centering
  \includegraphics[width=0.86\textwidth]{figure/task2/fig1_en_Indicator_Weights_Distribution.pdf}
  \caption{Indicator Weights Distribution (Entropy Weight Method)}
  \label{fig:Indicator_Weights_Distribution}
\end{figure}

\begin{table}[htbp]
\centering
\caption{Entropy Weights of Representative Indicators (Top and Bottom)}
\label{tab:entropy_weights_top_bottom}
\begin{tabular}{lccc}
\toprule
Indicator & Entropy $e_j$ & Redundancy $(1-e_j)$ & Weight $w_j$ \\
\midrule
Large Models & 0.2144 & 0.7856 & 0.0946 \\
International AI Investment & 0.2862 & 0.7138 & 0.0859 \\
Top AI Scholars & 0.4192 & 0.5808 & 0.0699 \\
AI Market Size & 0.4201 & 0.5799 & 0.0698 \\
AI Policies & 0.4444 & 0.5556 & 0.0669 \\
\midrule
AI Penetration & 0.9300 & 0.0700 & 0.0084 \\
5G Coverage & 0.9275 & 0.0725 & 0.0087 \\
Internet Penetration & 0.9478 & 0.0522 & 0.0063 \\
\bottomrule
\end{tabular}
\end{table}

\subsection{TOPSIS-Based Comprehensive Evaluation}

TOPSIS aggregates weighted indicators into a single competitiveness score by comparing each
country with an ideal reference. The method assumes that the optimal country should be closest
to the positive ideal solution and farthest from the negative ideal solution.

The weighted decision matrix is defined as
\begin{equation}
v_{ij}=w_j x'_{ij}.
\end{equation}
For each indicator $j$, the positive and negative ideal components are defined as
\begin{equation}
A_j^+=\max_{i} v_{ij},\quad A_j^-=\min_{i} v_{ij}, \qquad j=1,\dots,p,
\end{equation}
yielding the ideal vectors $A^+=(A_1^+,\dots,A_p^+)$ and $A^-=(A_1^-,\dots,A_p^-)$. \\

The distances from country $i$ to the ideal solutions are computed as
\begin{equation}
D_i^\pm=\sqrt{\sum_{j=1}^{p}(v_{ij}-A_j^\pm)^2},
\end{equation}
and the comprehensive competitiveness score is defined as
\begin{equation}
C_i=\frac{D_i^-}{D_i^++D_i^-}.
\end{equation}

\paragraph{Result interpretation.}
The entropy-weighted TOPSIS results demonstrate strong heterogeneity in national AI development capability in 2025.
The United States ranks first with $C=0.6407$, followed by China with $C=0.5104$,
forming a clear leading tier.
India ($C=0.2098$) and the UAE ($C=0.1836$ constitute the next tier, while the United Kingdom, South Korea, France, Japan,
and Germany cluster in a compact mid-lower band ($C\approx 0.062$--$0.100$).
Canada ranks last with $C=0.0414$.

This ``head--middle--tail'' pattern indicates that top countries outperform others simultaneously across multiple high-weight indicators,
while the separation among mid-ranked countries is relatively small and therefore sensitive to structural emphasis across dimensions.

\begin{figure}[htbp]
  \centering
  \includegraphics[width=0.86\textwidth]{figure/task2/fig2_en_AI_Competitiveness_Ranking_2025.pdf}
  \caption{AI Competitiveness Ranking in 2025 (TOPSIS Score)}
  \label{fig:AI_Competitiveness_Ranking_2025}
\end{figure}

\begin{table}[htbp]
\centering
\caption{TOPSIS Comprehensive Evaluation Results (2025)}
\label{tab:topsis_results_2025}
\begin{tabular}{lcccc}
\toprule
Country & $D_i^+$ & $D_i^-$ & TOPSIS Score $C_i$ & Rank \\
\midrule
United States & 0.0978 & 0.1744 & 0.6407 & 1 \\
China & 0.1320 & 0.1377 & 0.5104 & 2 \\
India & 0.2006 & 0.0533 & 0.2098 & 3 \\
UAE & 0.1998 & 0.0449 & 0.1836 & 4 \\
United Kingdom & 0.2040 & 0.0225 & 0.0995 & 5 \\
South Korea & 0.2063 & 0.0165 & 0.0740 & 6 \\
France & 0.2065 & 0.0153 & 0.0688 & 7 \\
Japan & 0.2064 & 0.0141 & 0.0640 & 8 \\
Germany & 0.2063 & 0.0138 & 0.0625 & 9 \\
Canada & 0.2091 & 0.0090 & 0.0414 & 10 \\
\bottomrule
\end{tabular}
\end{table}

\subsection{Structural Validation by Grey Relational Analysis}

Grey Relational Analysis is suitable for small-sample and partially known systems.
It evaluates the similarity of development patterns by comparing the geometric proximity
of indicator sequences.

Let the reference sequence be the ideal profile
\begin{equation}
v_{0j}=\max_{i} v_{ij}, \qquad j=1,\dots,p.
\end{equation}
Define the absolute deviation
\begin{equation}
\Delta_{ij}=|v_{0j}-v_{ij}|,
\end{equation}
and let
\begin{equation}
\Delta_{\min}=\min_{i,j}\Delta_{ij},\qquad
\Delta_{\max}=\max_{i,j}\Delta_{ij}.
\end{equation}
The grey relational coefficient is defined as
\begin{equation}
\xi_{ij}=\frac{\Delta_{\min}+\rho\Delta_{\max}}{\Delta_{ij}+\rho\Delta_{\max}}, \quad \rho=0.5,
\end{equation}
and the grey relational degree of country $i$ is calculated as
\begin{equation}
\gamma_i=\frac{1}{p}\sum_{j=1}^{p}\xi_{ij}.
\end{equation}

\paragraph{Result interpretation.}
GRA is employed as an independent structural validation of the TOPSIS-based ranking.
The resulting grey relational degrees preserve the same leading pair as TOPSIS:
the United States ranks first ($\gamma=0.7793$) and China ranks second ($\gamma=0.6440$).
India ($\gamma=0.4170$) and the UAE ($\gamma=0.4052$) remain in the upper-middle group.
Within the mid-ranked countries, slight differences emerge (e.g., South Korea ranks above the United Kingdom in GRA),
indicating that some countries may have more coherent indicator structures even when their absolute TOPSIS scores are close.

Overall, the close alignment between TOPSIS and GRA supports that the ranking is not an artifact of a single evaluation logic,
but reflects stable and systematic capability differences across countries.

\begin{table}[htbp]
\centering
\caption{Grey Relational Degrees and Rankings}
\label{tab:gra_results_2025}
\begin{tabular}{lcc}
\toprule
Country & Grey Relational Degree $\gamma_i$ & Rank \\
\midrule
United States & 0.7793 & 1 \\
China & 0.6440 & 2 \\
India & 0.4170 & 3 \\
UAE & 0.4052 & 4 \\
South Korea & 0.3703 & 5 \\
United Kingdom & 0.3686 & 6 \\
Japan & 0.3623 & 7 \\
France & 0.3612 & 8 \\
Germany & 0.3567 & 9 \\
Canada & 0.3506 & 10 \\
\bottomrule
\end{tabular}
\end{table}

\subsection{Fusion Ranking and Reliability Framework}

To integrate distance-based performance and structural similarity, an equal-weight fusion score
is defined as
\begin{equation}
S_i=\frac{C_i+\gamma_i}{2}.
\end{equation}
The fusion score $S_i$ is used to determine the final AI competitiveness ranking for 2025.

Ranking consistency is quantified using Spearman's rank correlation coefficient
\begin{equation}
\rho_s = 1-\frac{6\sum_{i=1}^{n} d_i^2}{n(n^2-1)},
\end{equation}
where $d_i$ denotes the rank difference of country $i$ between two ranking methods.
Robustness is further assessed by perturbing a single indicator weight $w_j$ by $\pm\alpha$
(e.g., $\alpha=30\%$), followed by weight renormalization and re-ranking.

\paragraph{Final ranking and reliability analysis.}
The equal-weight fusion of TOPSIS and GRA produces the final 2025 competitiveness ranking.
The top two countries (United States and China) remain unchanged, and most mid-ranked countries
show only minor rank adjustments, indicating a stable ordering.
Method consistency is extremely high: the rank correlation between TOPSIS scores and grey relational degrees
achieves Spearman $\rho_s=0.9758$ with statistical significance ($p<0.001$), confirming that the evaluation outcome
is robust to the choice of ranking mechanism.
Sensitivity results further indicate limited rank volatility under weight perturbations: most countries exhibit
rank ranges of 0--1, and only a few countries show moderate variation (e.g., France with range 3), suggesting
overall model stability.

\begin{figure}[htbp]
  \centering
  \includegraphics[width=0.86\textwidth]{figure/task2/fig7_en_Validation_Analysis.pdf}
  \caption{Validation and Robustness Analysis (Method Consistency and Sensitivity)}
  \label{fig:Validation_Analysis}
\end{figure}

\begin{table}[htbp]
\centering
\caption{Final AI Competitiveness Ranking (2025) by Fusion Score}
\label{tab:final_ranking_2025}
\begin{tabular}{lcccccc}
\toprule
Country & TOPSIS Rank & GRA Rank & TOPSIS $C_i$ & GRA $\gamma_i$ & Fusion Score $S_i$ & Grade \\
\midrule
United States & 1 & 1 & 0.6407 & 0.7793 & 0.7100 & A+ \\
China & 2 & 2 & 0.5104 & 0.6440 & 0.5772 & A \\
India & 3 & 3 & 0.2098 & 0.4170 & 0.3134 & A \\
UAE & 4 & 4 & 0.1836 & 0.4052 & 0.2944 & B \\
United Kingdom & 5 & 6 & 0.0995 & 0.3686 & 0.2341 & C \\
South Korea & 6 & 5 & 0.0740 & 0.3703 & 0.2221 & C \\
France & 7 & 8 & 0.0688 & 0.3612 & 0.2150 & C \\
Japan & 8 & 7 & 0.0640 & 0.3623 & 0.2131 & C \\
Germany & 9 & 9 & 0.0625 & 0.3567 & 0.2096 & C \\
Canada & 10 & 10 & 0.0414 & 0.3506 & 0.1960 & C \\
\bottomrule
\end{tabular}
\end{table}

\begin{table}[htbp]
\centering
\caption{Ranking Sensitivity under Weight Perturbation (Rank Range)}
\label{tab:sensitivity_range}
\begin{tabular}{lc}
\toprule
Country & Rank Range \\
\midrule
United States & 0 \\
China & 0 \\
India & 0 \\
UAE & 0 \\
United Kingdom & 0 \\
Canada & 0 \\
Germany & 1 \\
South Korea & 1 \\
Japan & 2 \\
France & 3 \\
\bottomrule
\end{tabular}
\end{table}

\paragraph{Structural explanation by six dimensions (optional).}
To interpret why top-ranked countries outperform others, the six-dimensional scores provide a compact decomposition.
The United States exhibits strong advantages in \textit{R\&D} ($0.8307$) and \textit{Infrastructure} ($0.8714$),
while China shows prominent strengths in \textit{Application} ($0.7344$) and \textit{Policy} ($0.7980$).
India displays relatively high \textit{Output} ($0.5889$) and moderate \textit{Application} ($0.4011$),
whereas the UAE relies more on \textit{Infrastructure} ($0.6453$) and policy-related support.
These complementary structures help explain why the leading tier remains stable while mid-ranked countries cluster closely.

\begin{figure}[htbp]
  \centering
  \includegraphics[width=0.80\textwidth]{figure/task2/fig4_en_Six_Dimensional_Capability_Radar.pdf}
  \caption{Six-Dimensional Capability Profiles of Representative Countries}
  \label{fig:Six_Dimensional_Radar}
\end{figure}

\subsection{Overall Summary of Task 2}

Task 2 establishes a comprehensive evaluation framework for national AI development capability
by integrating objective weighting, distance-based evaluation, and structural validation.
Using entropy-based weights, the TOPSIS model yields a clear 2025 competitiveness hierarchy,
with the United States and China forming the leading tier, followed by India and the UAE,
and a compact cluster of mid-ranked countries.

The GRA results provide an independent validation of the TOPSIS ordering, and the extremely high
rank correlation (Spearman $\rho_s=0.9758$, $p<0.001$) confirms strong methodological consistency.
Sensitivity analysis further indicates that rankings remain stable under reasonable perturbations of indicator weights.
Overall, the final fusion ranking offers a transparent and reproducible benchmark for 2025 AI competitiveness,
serving as the baseline for forecasting and policy analysis in subsequent tasks.
