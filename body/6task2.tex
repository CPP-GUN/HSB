\section{Task 2: Evaluation of National AI Competitiveness in 2025}

\subsection{Comprehensive Evaluation Methodology}

Based on the indicator system established in Task~1,
this task constructs an objective framework to evaluate national AI competitiveness in 2025.

Let the standardized indicator matrix be
\begin{equation}
X'=\left(x'_{ij}\right)_{n\times p}, \quad n=10,\; p=24,
\end{equation}
where all indicators are benefit-type variables.
The objective is to synthesize multidimensional information into a single, interpretable competitiveness score.

\textbf{(1) Entropy-based indicator weighting.}
Indicator importance is determined using the entropy weight method.
For indicator $j$,
\begin{equation}
p_{ij}=\frac{x'_{ij}}{\sum_{i=1}^{n}x'_{ij}}, \qquad 0\ln0:=0,
\end{equation}
\begin{equation}
e_j=-k\sum_{i=1}^{n}p_{ij}\ln p_{ij}, \quad k=\frac{1}{\ln n},
\end{equation}
and the entropy weight is
\begin{equation}
w_j=\frac{1-e_j}{\sum_{j=1}^{p}(1-e_j)}.
\end{equation}
Indicators with greater cross-country dispersion receive higher weights.

The resulting weight distribution is shown in Figure~\ref{fig:Indicator_Weights_Distribution},
with representative indicators summarized in Table~\ref{tab:entropy_weights_top_bottom}.

\begin{figure}[htbp]
  \centering
  \includegraphics[width=0.75\textwidth]{figure/task2/fig1_en_Indicator_Weights_Distribution.pdf}
  \caption{Indicator Weights Distribution (Entropy Weight Method)}
  \label{fig:Indicator_Weights_Distribution}
\end{figure}

\begin{table}[htbp]
\centering
\caption{Entropy Weights of Representative Indicators (Top and Bottom)}
\label{tab:entropy_weights_top_bottom}
\begin{tabular}{lccc}
\toprule
Indicator & Entropy $e_j$ & Redundancy $(1-e_j)$ & Weight $w_j$ \\
\midrule
Large Models & 0.2144 & 0.7856 & 0.0946 \\
International AI Investment & 0.2862 & 0.7138 & 0.0859 \\
Top AI Scholars & 0.4192 & 0.5808 & 0.0699 \\
AI Market Size & 0.4201 & 0.5799 & 0.0698 \\
AI Policies & 0.4444 & 0.5556 & 0.0669 \\
\midrule
AI Penetration & 0.9300 & 0.0700 & 0.0084 \\
5G Coverage & 0.9275 & 0.0725 & 0.0087 \\
Internet Penetration & 0.9478 & 0.0522 & 0.0063 \\
\bottomrule
\end{tabular}
\end{table}

\textbf{(2) TOPSIS-based comprehensive evaluation.}
Using the entropy weights, the weighted decision matrix is
\begin{equation}
v_{ij}=w_j x'_{ij}.
\end{equation}
The positive and negative ideal solutions are
\begin{equation}
A_j^+=\max_i v_{ij}, \quad A_j^-=\min_i v_{ij},
\end{equation}
and the distances to these benchmarks are
\begin{equation}
D_i^\pm=\sqrt{\sum_{j=1}^{p}(v_{ij}-A_j^\pm)^2}.
\end{equation}
The TOPSIS score is defined as
\begin{equation}
C_i=\frac{D_i^-}{D_i^++D_i^-}.
\end{equation}

\textbf{(3) Structural validation by grey relational analysis.}
Let the ideal profile be
\begin{equation}
v_{0j}=\max_i v_{ij}.
\end{equation}
Define
\begin{equation}
\Delta_{ij}=|v_{0j}-v_{ij}|,\quad
\Delta_{\min}=\min_{i,j}\Delta_{ij},\quad
\Delta_{\max}=\max_{i,j}\Delta_{ij}.
\end{equation}
The grey relational coefficient and degree are
\begin{equation}
\xi_{ij}=\frac{\Delta_{\min}+0.5\,\Delta_{\max}}{\Delta_{ij}+0.5\,\Delta_{\max}},
\qquad
\gamma_i=\frac{1}{p}\sum_{j=1}^{p}\xi_{ij}.
\end{equation}

\subsection{Results and Comparative Analysis}

\textbf{(1) Indicator weight structure.}
Entropy weights concentrate on frontier and output-oriented indicators,
indicating that cross-country differences are primarily driven by advanced innovation capacity
and global resource integration, while basic digital penetration indicators contribute marginally.

\textbf{(2) National competitiveness ranking in 2025.}
The TOPSIS results are reported in Table~\ref{tab:topsis_results_2025}
and Figure~\ref{fig:AI_Competitiveness_Ranking_2025}.
The United States and China form a clear leading tier,
followed by India and the UAE, while the remaining countries cluster closely.

\begin{figure}[htbp]
  \centering
  \includegraphics[width=0.75\textwidth]{figure/task2/fig2_en_AI_Competitiveness_Ranking_2025.pdf}
  \caption{AI Competitiveness Ranking in 2025 (TOPSIS Score)}
  \label{fig:AI_Competitiveness_Ranking_2025}
\end{figure}

\begin{table}[htbp]
\centering
\caption{TOPSIS Comprehensive Evaluation Results (2025)}
\label{tab:topsis_results_2025}
\begin{tabular}{lcccc}
\toprule
Country & $D_i^+$ & $D_i^-$ & TOPSIS Score $C_i$ & Rank \\
\midrule
United States & 0.0978 & 0.1744 & 0.6407 & 1 \\
China & 0.1320 & 0.1377 & 0.5104 & 2 \\
India & 0.2006 & 0.0533 & 0.2098 & 3 \\
UAE & 0.1998 & 0.0449 & 0.1836 & 4 \\
United Kingdom & 0.2040 & 0.0225 & 0.0995 & 5 \\
South Korea & 0.2063 & 0.0165 & 0.0740 & 6 \\
France & 0.2065 & 0.0153 & 0.0688 & 7 \\
Japan & 0.2064 & 0.0141 & 0.0640 & 8 \\
Germany & 0.2063 & 0.0138 & 0.0625 & 9 \\
Canada & 0.2091 & 0.0090 & 0.0414 & 10 \\
\bottomrule
\end{tabular}
\end{table}

\textbf{(3) Cross-method validation and fusion ranking.}
The grey relational results are summarized in Table~\ref{tab:gra_results_2025}.
To integrate distance-based performance and structural similarity,
the fusion score is defined as
\begin{equation}
S_i=\frac{C_i+\gamma_i}{2}.
\end{equation}
The final ranking is reported in Table~\ref{tab:final_ranking_2025}.

\begin{table}[htbp]
\centering
\caption{Grey Relational Degrees and Rankings}
\label{tab:gra_results_2025}
\begin{tabular}{lcc}
\toprule
Country & Grey Relational Degree $\gamma_i$ & Rank \\
\midrule
United States & 0.7793 & 1 \\
China & 0.6440 & 2 \\
India & 0.4170 & 3 \\
UAE & 0.4052 & 4 \\
South Korea & 0.3703 & 5 \\
United Kingdom & 0.3686 & 6 \\
Japan & 0.3623 & 7 \\
France & 0.3612 & 8 \\
Germany & 0.3567 & 9 \\
Canada & 0.3506 & 10 \\
\bottomrule
\end{tabular}
\end{table}

\begin{table}[htbp]
\centering
\caption{Final AI Competitiveness Ranking (2025) by Fusion Score}
\label{tab:final_ranking_2025}
\begin{tabular}{lcccccc}
\toprule
Country & TOPSIS Rank & GRA Rank & TOPSIS $C_i$ & GRA $\gamma_i$ & Fusion Score $S_i$ & Grade \\
\midrule
United States & 1 & 1 & 0.6407 & 0.7793 & 0.7100 & A+ \\
China & 2 & 2 & 0.5104 & 0.6440 & 0.5772 & A \\
India & 3 & 3 & 0.2098 & 0.4170 & 0.3134 & A \\
UAE & 4 & 4 & 0.1836 & 0.4052 & 0.2944 & B \\
United Kingdom & 5 & 6 & 0.0995 & 0.3686 & 0.2341 & C \\
South Korea & 6 & 5 & 0.0740 & 0.3703 & 0.2221 & C \\
France & 7 & 8 & 0.0688 & 0.3612 & 0.2150 & C \\
Japan & 8 & 7 & 0.0640 & 0.3623 & 0.2131 & C \\
Germany & 9 & 9 & 0.0625 & 0.3567 & 0.2096 & C \\
Canada & 10 & 10 & 0.0414 & 0.3506 & 0.1960 & C \\
\bottomrule
\end{tabular}
\end{table}

\textbf{(4) Reliability and robustness assessment.}
Consistency between TOPSIS and GRA is measured using Spearman’s rank correlation coefficient,
yielding $\rho_s=0.9758$ ($p<0.001$).
Sensitivity analysis under $\pm30\%$ weight perturbations confirms that most countries
exhibit rank variations of at most one position.
The validation results are illustrated in Figure~\ref{fig:Validation_Analysis}
and summarized in Table~\ref{tab:sensitivity_range}.

\begin{figure}[htbp]
  \centering
  \includegraphics[width=0.75\textwidth]{figure/task2/fig7_en_Validation_Analysis.pdf}
  \caption{Validation and Robustness Analysis (Method Consistency and Sensitivity)}
  \label{fig:Validation_Analysis}
\end{figure}

\begin{table}[htbp]
\centering
\caption{Ranking Sensitivity under Weight Perturbation (Rank Range)}
\label{tab:sensitivity_range}
\begin{tabular}{lc}
\toprule
Country & Rank Range \\
\midrule
United States & 0 \\
China & 0 \\
India & 0 \\
UAE & 0 \\
United Kingdom & 0 \\
Canada & 0 \\
Germany & 1 \\
South Korea & 1 \\
Japan & 2 \\
France & 3 \\
\bottomrule
\end{tabular}
\end{table}
