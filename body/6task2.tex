\section{Task 2: AI Development Capability Evaluation and 2025 Ranking}

\subsection{Model Overview}

Based on the 24 indicators identified in Task 1, let the normalized indicator matrix be
\begin{equation}
X'=\left(x'_{ij}\right)_{n\times p},\quad n=10,\;p=24,
\end{equation}
where $x'_{ij}$ denotes the normalized value of indicator $j$ for country $i$.
All indicators have been unified as benefit-type (larger values indicate stronger AI development capability)
and normalized prior to this task.

The objective of Task 2 is to construct an \textit{objective and reproducible} evaluation model
to quantify national AI development capability and determine the 2025 competitiveness ranking
of ten countries.

To reduce subjective bias and enhance robustness, an integrated evaluation framework combining
the Entropy Weight Method (EWM), TOPSIS, and Grey Relational Analysis (GRA) is adopted.

\subsection{Entropy-Based Weighting}

The entropy weight method is derived from information theory and assigns indicator weights
according to their dispersion across countries. Indicators with higher variability contain
more effective information for discrimination and thus receive larger weights.

Define
\begin{equation}
p_{ij}=\frac{x'_{ij}}{\sum_{i=1}^{n}x'_{ij}},\qquad 0\ln 0 := 0,
\end{equation}
where $p_{ij}$ represents the proportion of indicator $j$ contributed by country $i$.
The information entropy of indicator $j$ is computed as
\begin{equation}
e_j=-k\sum_{i=1}^{n}p_{ij}\ln p_{ij}, \quad k=\frac{1}{\ln n}.
\end{equation}
The corresponding entropy weight is given by
\begin{equation}
w_j=\frac{1-e_j}{\sum_{j=1}^{p}(1-e_j)}.
\end{equation}
If $\sum_{i=1}^{n}x'_{ij}=0$ for some indicator $j$, its weight is set to zero and the remaining
weights are renormalized.

\textit{(Result observation and interpretation to be added here.)}

\subsection{TOPSIS-Based Comprehensive Evaluation}

TOPSIS aggregates weighted indicators into a single competitiveness score by comparing each
country with an ideal reference. The method assumes that the optimal country should be closest
to the positive ideal solution and farthest from the negative ideal solution.

The weighted decision matrix is defined as
\begin{equation}
v_{ij}=w_j x'_{ij}.
\end{equation}
For each indicator $j$, the positive and negative ideal components are defined as
\begin{equation}
A_j^+=\max_{i} v_{ij},\quad A_j^-=\min_{i} v_{ij}, \qquad j=1,\dots,p,
\end{equation}
yielding the ideal vectors $A^+=(A_1^+,\dots,A_p^+)$ and $A^-=(A_1^-,\dots,A_p^-)$.

The distances from country $i$ to the ideal solutions are computed as
\begin{equation}
D_i^\pm=\sqrt{\sum_{j=1}^{p}(v_{ij}-A_j^\pm)^2},
\end{equation}
and the comprehensive competitiveness score is defined as
\begin{equation}
C_i=\frac{D_i^-}{D_i^++D_i^-}.
\end{equation}

\textit{(Result observation and interpretation to be added here.)}

\subsection{Structural Validation by Grey Relational Analysis}

Grey Relational Analysis is suitable for small-sample and partially known systems.
It evaluates the similarity of development patterns by comparing the geometric proximity
of indicator sequences.

Let the reference sequence be the ideal profile
\begin{equation}
v_{0j}=\max_{i} v_{ij}, \qquad j=1,\dots,p.
\end{equation}
Define the absolute deviation
\begin{equation}
\Delta_{ij}=|v_{0j}-v_{ij}|,
\end{equation}
and let
\begin{equation}
\Delta_{\min}=\min_{i,j}\Delta_{ij},\qquad
\Delta_{\max}=\max_{i,j}\Delta_{ij}.
\end{equation}
The grey relational coefficient is defined as
\begin{equation}
\xi_{ij}=\frac{\Delta_{\min}+\rho\Delta_{\max}}{\Delta_{ij}+\rho\Delta_{\max}}, \quad \rho=0.5,
\end{equation}
and the grey relational degree of country $i$ is calculated as
\begin{equation}
\gamma_i=\frac{1}{p}\sum_{j=1}^{p}\xi_{ij}.
\end{equation}

\textit{(Result observation and interpretation to be added here.)}

\subsection{Fusion Ranking and Reliability Framework}

To integrate distance-based performance and structural similarity, an equal-weight fusion score
is defined as
\begin{equation}
S_i=\frac{C_i+\gamma_i}{2}.
\end{equation}
The fusion score $S_i$ is used to determine the final AI competitiveness ranking for 2025.

Ranking consistency is quantified using Spearman's rank correlation coefficient
\begin{equation}
\rho_s = 1-\frac{6\sum_{i=1}^{n} d_i^2}{n(n^2-1)},
\end{equation}
where $d_i$ denotes the rank difference of country $i$ between two ranking methods.
Robustness is further assessed by perturbing a single indicator weight $w_j$ by $\pm\alpha$
(e.g., $\alpha=30\%$), followed by weight renormalization and re-ranking.

\textit{(Final ranking results and reliability analysis to be added here.)}

\subsection{Overall Summary of Task 2}

Task 2 establishes a comprehensive evaluation framework for national AI development capability
by integrating objective weighting, distance-based evaluation, and structural validation.
The framework provides a consistent and reproducible basis for generating the 2025 competitiveness
ranking, which serves as the benchmark for subsequent analysis.

\textit{(Overall result summary and interpretation to be added here.)}
