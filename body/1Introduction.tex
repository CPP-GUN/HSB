\section{Intorduction}

\subsection{Background}

In the contemporary era, artificial intelligence (AI) has emerged as one of the core domains of global technological competition, exerting profound and systemic influences on economic development, social progress, and national security. With the acceleration of a new wave of technological revolution and industrial transformation, AI technologies are fundamentally reshaping traditional industrial structures, modes of production, and governance systems, and have gradually become a key indicator of a nation’s scientific strength and overall competitiveness.

Against this backdrop, countries around the world have elevated artificial intelligence to a strategic priority at the national level, continuously increasing investments in algorithmic research, computing infrastructure, data resource development, and the expansion of application scenarios, with the aim of securing a leading position in the global AI competitive landscape.

\subsection{Problem Restatement}

This study aims to quantitatively evaluate national artificial intelligence (AI) development capabilities, compare global competitiveness, and analyze future development trends through a systematic mathematical modeling framework. The problem is decomposed into four sequential and interrelated tasks:

\textbf{Task 1: Factor Identification and Correlation Analysis}

Relevant data are collected and integrated to identify the key factors influencing national AI development. These factors are quantified, and their intrinsic correlations and interaction mechanisms are analyzed using statistical and visualization methods.

\textbf{Task 2: Comprehensive Evaluation and Ranking}

Based on the quantified factors and their correlations obtained in Task 1, a multi-criteria evaluation model is constructed to assess and rank the AI competitiveness of ten selected countries.

\textbf{Task 3: Competitiveness Trend Prediction}

Using historical data from 2016 to 2025, the future evolution of AI development factors during the period 2026–2035 is predicted. The evaluation model established in Task 2 is then applied to analyze the dynamic changes in national competitiveness rankings over time.

\textbf{Task 4: Optimal Fund Allocation Strategy}

Under a fixed budget constraint of a 1 trillion yuan special fund, a multi-objective optimization model is developed to determine the optimal allocation of resources across AI development factors, with the goal of maximizing China’s comprehensive AI competitiveness by 2035.

By sequentially accomplishing these tasks, this study provides a coherent framework for factor identification, comparative evaluation, future trend analysis, and strategic decision support in the global AI competition landscape.

\subsection{Our Work}

这是我们的工作介绍
