%%%%%%%%%%%%%%%%%%%%%%%%%%%%%%%%%%%%%%%%
%% MCM/ICM LaTeX Template %%
%% 2026 MCM/ICM           %%
%%%%%%%%%%%%%%%%%%%%%%%%%%%%%%%%%%%%%%%%
\documentclass[12pt]{article}
\usepackage{geometry}
\geometry{left=1in,right=1in,top=1in,bottom=1in}

%%%%%%%%%%%%%%%%%%%%%%%%%%%%%%%%%%%%%%%%

\renewcommand{\topfraction}{0.9}
\renewcommand{\bottomfraction}{0.9}
\renewcommand{\textfraction}{0.05}
\renewcommand{\floatpagefraction}{0.8}

\setcounter{topnumber}{5}
\setcounter{bottomnumber}{5}
\setcounter{totalnumber}{10}

\newcommand{\Problem}{B}
\newcommand{\Team}{2601258}

%%%%%%%%%%%%%%%%%%%%%%%%%%%%%%%%%%%%%%%%

\usepackage{placeins}
\usepackage{float}

\usepackage{newtxtext}
\usepackage{amsmath,amssymb,amsthm}
\usepackage{newtxmath}
\usepackage{lipsum}
\usepackage{booktabs}
\usepackage{multirow}
\usepackage{array}
\usepackage{indentfirst}
\usepackage{longtable}
\usepackage{needspace}
\usepackage{subcaption}

\usepackage[pdftex]{graphicx}
\usepackage{xcolor}
\usepackage{fancyhdr}
\usepackage{lastpage}
\usepackage{tocloft}
\renewcommand{\cftsecleader}{\cftdotfill{\cftdotsep}}
\usepackage[hidelinks]{hyperref}

\newtheorem{theorem}{Theorem}
\newtheorem{corollary}[theorem]{Corollary}
\newtheorem{lemma}[theorem]{Lemma}
\newtheorem{definition}{Definition}

\fancypagestyle{plain}{
	\fancyhf{}
	\lhead{Team \# \Team}
	\rhead{Page \thepage~of \pageref{LastPage}}
	%\rhead{Page \thepage~of 25}
	\cfoot{}
}

%%%%%%%%%%%%%%%%%%%%%%%%%%%%%%%%

\begin{document}

%\begin{CJK}{UTF8}{gbsn}

\graphicspath{{.}}
\DeclareGraphicsExtensions{.pdf, .jpg, .tif, .png}
\thispagestyle{empty}
\vspace*{-16ex}
\centerline{\begin{tabular}{*3{c}}
		\parbox[t]{0.3\linewidth}{\begin{center}\textbf{Problem Chosen}\\ \Large \textcolor{red}{\Problem}\end{center}}
		 & \parbox[t]{0.3\linewidth}{\begin{center}\textbf{2026HSB\\ MCM/ICM\\ Summary Sheet}\end{center}}
		 & \parbox[t]{0.3\linewidth}{\begin{center}\textbf{Team  Number}\\ \Large \textcolor{red}{\Team}\end{center}} \\
		\hline
	\end{tabular}}

%%%%%%%%%%% Begin Summary %%%%%%%%%%%
\begin{center}
	\LARGE {Assessment, Forecasting, and Optimization of National AI Competitiveness}\\
	\vspace{0.4cm}
	\normalsize\textbf{Summary}
\end{center}

Artificial intelligence has become a core driver of national competitiveness. This paper develops a unified framework to evaluate, forecast, and optimize national AI capabilities based on \textbf{24 indicators} from \textbf{10 countries} (2016--2025).

\textbf{In Problem 1}, Pearson correlation analysis identifies \textbf{86 strongly correlated} indicator pairs ($|r|>0.7$). Principal Component Analysis extracts \textbf{four components} explaining \textbf{85\%} cumulative variance, revealing \textbf{infrastructure}, \textbf{talent}, and \textbf{innovation} as key drivers.

\textbf{For Problem 2}, we employ the \textbf{Entropy Weight Method} and \textbf{TOPSIS} to rank 2025 AI competitiveness: \textbf{United States} first (0.641), \textbf{China} second (0.510), \textbf{India} third (0.210). \textbf{Grey Relational Analysis} validates the ranking with Spearman correlation $\rho=0.95$.

\textbf{Problem 3} applies \textbf{GM(1,1)} grey forecasting to project 2026--2035 trends for all 24 indicators. Backtest diagnostics show median \textbf{MAPE} $10.35\%$. Results indicate top-three countries maintain structural advantages over the decade.

\textbf{In Problem 4}, \textbf{Sequential Least Squares Programming (SLSQP)} optimizes China's \textbf{CNY 1 trillion} investment allocation: \textbf{infrastructure} 32.33\%, \textbf{talent cultivation} 17.39\%, \textbf{policy support} 17.39\%. Sensitivity analysis confirms robustness under weight perturbation.

The framework integrates evaluation, forecasting, and optimization with multi-layer validation, providing \textbf{reproducible decision support} for national AI strategic planning.

\vspace{0.4cm}
\noindent \textbf{Keywords:} AI Development Capability, TOPSIS, Grey Forecasting, Entropy Weighting, Investment Optimization, Competitiveness Ranking
%%%%%%%%%%% End Summary %%%%%%%%%%%

\clearpage
\pagestyle{plain}
\newpage
\setcounter{page}{1}
%%%%%%%%%%%%%%%%%%%%%%%%%%%%%%
%目录
\tableofcontents
\newpage

%正文
\section{Intorduction}

\subsection{Background}

In the contemporary era, artificial intelligence (AI) has emerged as one of the core domains of global technological competition, exerting profound and systemic influences on economic development, social progress, and national security. With the acceleration of a new wave of technological revolution and industrial transformation, AI technologies are fundamentally reshaping traditional industrial structures, modes of production, and governance systems, and have gradually become a key indicator of a nation’s scientific strength and overall competitiveness.

Against this backdrop, countries around the world have elevated artificial intelligence to a strategic priority at the national level, continuously increasing investments in algorithmic research, computing infrastructure, data resource development, and the expansion of application scenarios, with the aim of securing a leading position in the global AI competitive landscape.

\subsection{Problem Restatement}

This study aims to quantitatively evaluate national artificial intelligence (AI) development capabilities, compare global competitiveness, and analyze future development trends through a systematic mathematical modeling framework. The problem is decomposed into four sequential and interrelated tasks:

\textbf{Task 1: Factor Identification and Correlation Analysis}

Relevant data are collected and integrated to identify the key factors influencing national AI development. These factors are quantified, and their intrinsic correlations and interaction mechanisms are analyzed using statistical and visualization methods.

\textbf{Task 2: Comprehensive Evaluation and Ranking}

Based on the quantified factors and their correlations obtained in Task 1, a multi-criteria evaluation model is constructed to assess and rank the AI competitiveness of ten selected countries.

\textbf{Task 3: Competitiveness Trend Prediction}

Using historical data from 2016 to 2025, the future evolution of AI development factors during the period 2026–2035 is predicted. The evaluation model established in Task 2 is then applied to analyze the dynamic changes in national competitiveness rankings over time.

\textbf{Task 4: Optimal Fund Allocation Strategy}

Under a fixed budget constraint of a 1 trillion yuan special fund, a multi-objective optimization model is developed to determine the optimal allocation of resources across AI development factors, with the goal of maximizing China’s comprehensive AI competitiveness by 2035.

By sequentially accomplishing these tasks, this study provides a coherent framework for factor identification, comparative evaluation, future trend analysis, and strategic decision support in the global AI competition landscape.

\subsection{Our Work}

这是我们的工作介绍

\section{Basic Assumption}

To ensure the feasibility, consistency, and interpretability of the proposed models, the following basic assumptions are made.

$\blacktriangleright$ \textbf{Hypothesis 1: Assume that national AI development capability is a latent attribute that can be approximated by a finite set of observable and quantifiable indicators.}

\textbf{Legitimacy: }At the national level, AI development is manifested through measurable outcomes and resource inputs recorded in public statistics. Although the true capability cannot be observed directly, its major characteristics can be reasonably inferred from aggregated, quantifiable indicators.

$\blacktriangleright$ \textbf{Hypothesis 2: Assume that all indicators within the same evaluation year are cross-sectionally consistent.}

\textbf{Legitimacy: }Although data may be collected from slightly different release years, AI development is a long-term process. Minor temporal discrepancies do not significantly affect national-level competitiveness comparisons and help simplify the modeling process.

$\blacktriangleright$ \textbf{Hypothesis 3: Assume that the indicators are independent of each other in the weighting and evaluation stages.}

\textbf{Justification :}While interactions among indicators exist, explicitly modeling such dependencies would increase complexity and reduce interpretability. Treating indicators as independent avoids double counting and ensures the applicability of entropy-based and multi-criteria evaluation methods.

$\blacktriangleright$ \textbf{Hypothesis 4: Assume that the fundamental mechanisms of AI development remain stable during the forecasting and optimization period.}

\textbf{Justification :}National AI strategies, infrastructure construction, and talent cultivation generally evolve gradually. This stability makes trend-based prediction and investment optimization reasonable and analytically tractable.

\section{Symbols}

\begin{table}[h]
\centering
\caption{Notation and Symbol Definitions}
\label{tab:symbols}
\begin{tabular}{cl}
\toprule
\textbf{Symbol} & \textbf{Definition} \\
\midrule
% Basic parameters used across all tasks
$n$ & Number of countries (samples) \\
$p$ & Number of indicators \\
% Task 1 symbols: Factor Identification and Correlation Analysis
$x_{ij}$ & Raw value of indicator $j$ for country $i$ \\
$I_j$ & Comprehensive importance score of indicator $j$ \\
% Task 2 symbols: Comprehensive Evaluation and Ranking
$w_j$ & Entropy weight of indicator $j$ \\
$C_i$ & TOPSIS closeness score for country $i$ \\
% Task 3 symbols: Competitiveness Trend Prediction
$\hat{x}_{i,j,t}$ & Predicted value of indicator $j$ for country $i$ in year $t$ \\
$C_{i,t}$ & AI competitiveness score of country $i$ in year $t$ \\
% Task 4 symbols: Optimal Fund Allocation Strategy
$I_j$ & Investment allocation for indicator $j$ \\
$\Delta x_j(\mathbf{I})$ & Growth increment of indicator $j$ under investment $\mathbf{I}$ \\
\bottomrule
\end{tabular}
\end{table}


\section{Data Explanation}

\subsection{Data Description and Sources}

The dataset covers ten representative countries and consists of multiple quantitative indicators describing national AI development capability. For organizational clarity, the indicators were grouped into six dimensions reflecting different aspects of AI development. All data corresponded to the same evaluation period and were obtained from publicly available and widely recognized sources, ensuring cross-country comparability.

\subsection{Data Preprocessing}

Basic preprocessing was conducted prior to analysis. Minor missing or abnormal values were handled through reasonable estimation and consistency checks. All indicators were defined as benefit-type variables and were normalized to eliminate dimensional differences before being used in subsequent models.

\section{Task 1: Identification and Structural Analysis of AI Development Factors}

\subsection{Factor System and Quantification}

To quantitatively characterize national AI development capability, a multi-factor indicator system is constructed and organized into a standardized data matrix. Let
\begin{equation}
X = [x_{ij}] \in \mathbb{R}^{n \times p}
\end{equation}
denote the normalized indicator matrix, where $x_{ij}$ represents the standardized value of indicator $j$ for country $i$, with $n=10$ countries and $p=24$ indicators.

To eliminate scale effects and ensure cross-country comparability, min--max normalization is applied:
\begin{equation}
x_{ij} = \frac{x_{ij}^{\text{raw}}-\min(x_j)}{\max(x_j)-\min(x_j)} \in [0,1].
\end{equation}
The standardized matrix $X$ serves as the common input for all subsequent structural analyses.

\subsection{Correlation Structure among AI Development Factors}

To explore linear associations among AI development factors, Pearson correlation coefficients are employed to quantify statistical dependence between indicators.

The correlation coefficient between factor $j$ and factor $k$ is defined as
\begin{equation}
r_{jk}=
\frac{\sum_{i=1}^{n}(x_{ij}-\bar{x}_j)(x_{ik}-\bar{x}_k)}
{\sqrt{\sum_{i=1}^{n}(x_{ij}-\bar{x}_j)^2
\sum_{i=1}^{n}(x_{ik}-\bar{x}_k)^2}},
\end{equation}
forming the correlation matrix
\begin{equation}
R=[r_{jk}] \in \mathbb{R}^{p \times p}.
\end{equation}

To emphasize statistically significant relationships, a strong-correlation edge set is defined as
\begin{equation}
\mathcal{E}=\{(j,k)\mid |r_{jk}|>\tau,\ j<k\},
\end{equation}
where $\tau$ denotes a predefined threshold.

The correlation results reveal a highly interconnected structure among AI development factors, with many indicator pairs exceeding the predefined threshold. This indicates pronounced co-movement rather than independent variation across dimensions.

As illustrated in the correlation heatmap, strong positive correlations form several contiguous blocks. Indicators related to government support, market scale, and investment intensity are closely linked with computing infrastructure and data-related factors, suggesting coordinated national development patterns. Talent-related indicators, including AI researchers, top AI scholars, and AI graduates, also exhibit strong mutual correlations and are closely associated with research output and enterprise activity.

By contrast, basic infrastructure penetration indicators display weaker correlations with advanced AI capability measures, implying that foundational infrastructure alone does not fully explain cross-country differences. Overall, the dense correlation structure highlights system-level coupling across multiple dimensions and suggests potential redundancy among indicators, motivating further structural analysis.

\begin{figure}[htbp]
  \centering
  \includegraphics[width=0.8\textwidth]{figure/task1/fig1_en_Correlation Heatmap of AI Development Factors.pdf}
  \caption{Correlation Heatmap of AI Development Factors}
  \label{fig:Correlation Heatmap of AI Development Factors}
\end{figure}

\subsection{Structural Grouping via Hierarchical Clustering}

To uncover higher-level structural organization beyond pairwise correlations, hierarchical clustering is conducted based on correlation-derived distances. The distance between factor $j$ and factor $k$ is defined as
\begin{equation}
d_{jk}=1-|r_{jk}|,
\end{equation}
yielding the distance matrix $D=[d_{jk}]$.

Using the average linkage criterion, the distance between clusters $C_a$ and $C_b$ is computed as
\begin{equation}
D(C_a,C_b)=\frac{1}{|C_a||C_b|}\sum_{j\in C_a}\sum_{k\in C_b} d_{jk}.
\end{equation}

The hierarchical clustering dendrogram reveals several coherent groups of AI development factors. Rather than being randomly aggregated, indicators are organized into structurally meaningful clusters reflecting consistent cross-country behavior.

One cluster primarily captures investment scale and market outcomes, including government and international AI investment, AI market size, and corporate R\&D expenditure. Another cluster centers on computational capacity and infrastructure, grouping indicators such as GPU cluster scale, internet bandwidth, and data center availability. Talent- and knowledge-related indicators form a distinct cluster, reflecting strong alignment among human capital and research output factors.

In contrast, basic digital penetration indicators merge with other clusters at higher distance levels, indicating weaker direct association with advanced AI capability drivers. These results confirm that AI development factors form a multi-dimensional yet internally coordinated system.

\begin{figure}[htbp]
  \centering
  \includegraphics[width=0.8\textwidth]{figure/task1/fig2_en_Hierarchical Clustering Dendrogram of AI Development Factors.pdf}
  \caption{Hierarchical Clustering Dendrogram of AI Development Factors}
  \label{fig:Hierarchical Clustering Dendrogram of AI Development Factors}
\end{figure}

\subsection{Principal Component Analysis of Factor Structure}

Given the high dimensionality of the indicator set, principal component analysis (PCA) is applied to extract a reduced set of comprehensive components while preserving dominant structural information.

The centered data matrix is
\begin{equation}
\tilde{X}=X-\mathbf{1}\bar{X}^T,
\end{equation}
and the covariance matrix is
\begin{equation}
C=\frac{1}{n-1}\tilde{X}^T\tilde{X}.
\end{equation}

Eigen-decomposition yields
\begin{equation}
C=V\Lambda V^T,
\end{equation}
where $\Lambda=\mathrm{diag}(\lambda_1,\ldots,\lambda_p)$. The number of retained components $m$ satisfies
\begin{equation}
\sum_{k=1}^{m}\frac{\lambda_k}{\sum_{j=1}^{p}\lambda_j}\ge \eta.
\end{equation}

The PCA results show that variance is highly concentrated in the leading components. The first principal component captures a dominant share of total variance, indicating a strong common structural dimension across indicators. The cumulative variance exceeds 90\% after four components, while additional components contribute only marginal explanatory power.

These findings demonstrate substantial redundancy in the original indicator set and justify representing the system with a low-dimensional structure. Retaining four principal components preserves essential information while significantly reducing dimensional complexity.

\begin{figure}[htbp]
  \centering
  \includegraphics[width=0.8\textwidth]{figure/task1/fig3_en_Variance Explained Plot for Principal Components.pdf}
  \caption{Variance Explained Plot for Principal Components}
  \label{fig:Variance Explained Plot for Principal Components}
\end{figure}

\subsection{Relative Importance of AI Development Factors}

Building on the reduced PCA representation, the relative importance of each indicator is quantified by integrating component loadings and variance contributions. The importance of factor $j$ is defined as
\begin{equation}
I_j=\sum_{k=1}^{m} v_{jk}^2\cdot\frac{\lambda_k}{\sum_{l=1}^{p}\lambda_l}.
\end{equation}

The importance ranking indicates that explanatory power is concentrated in a limited subset of indicators. Factors related to human capital, research activity, investment intensity, and advanced infrastructure consistently exhibit higher importance scores.

High-importance indicators load strongly on the leading principal component, which dominates total variance. In contrast, lower-importance indicators primarily contribute to higher-order components and capture more localized variation. This concentration highlights the dominant drivers of cross-country AI capability differences and supports the use of weighted composite evaluation models.

\begin{figure}[htbp]
  \centering
  \includegraphics[width=0.8\textwidth]{figure/task1/fig4_en_Factor Importance Ranking Bar Chart.pdf}
  \caption{Factor Importance Ranking Bar Chart}
  \label{fig:Factor Importance Ranking Bar Chart}
\end{figure}

\subsection{Interaction Patterns among AI Development Factors}

The strong-correlation network reveals a centralized and modular interaction structure among AI development factors. Indicators related to investment, computational capacity, and market scale occupy hub positions, linking multiple dimensions of the system.

Peripheral indicators, such as social perception and basic penetration measures, exhibit fewer strong connections, indicating weaker coupling with core AI capability drivers. Overall, the network structure confirms that AI development capability emerges from coordinated interactions among a limited set of dominant factors rather than isolated contributions.

\begin{figure}[htbp]
  \centering
  \includegraphics[width=0.8\textwidth]{figure/task1/fig5_en_Strong-Correlation Network or Chord Diagram.pdf}
  \caption{Strong-Correlation Network or Chord Diagram}
  \label{fig:Strong-Correlation Network or Chord Diagram}
\end{figure}

\subsection{Task 1 Summary}

Task~1 provides a unified, data-driven structural analysis of national AI development capability by integrating correlation analysis, hierarchical clustering, PCA, and network analysis. The results demonstrate that AI development factors are highly interconnected and organized into coherent structural modules.

Dimensionality reduction reveals that a small number of dominant components capture most of the system’s variance, while factor importance analysis identifies key indicators driving cross-country differences. Together, these findings establish a compact and consistent analytical foundation for the subsequent evaluation, ranking, and forecasting tasks.

\section{Task 2: AI Development Capability Evaluation and 2025 Ranking}

\subsection{Model Overview}

Based on the 24 indicators identified in Task 1, let the normalized indicator matrix be
\begin{equation}
X'=\left(x'_{ij}\right)_{n\times p},\quad n=10,\;p=24,
\end{equation}
where $x'_{ij}$ denotes the normalized value of indicator $j$ for country $i$.
All indicators have been unified as benefit-type (larger values indicate stronger AI development capability)
and normalized prior to this task.

The objective of Task 2 is to construct an \textit{objective and reproducible} evaluation model
to quantify national AI development capability and determine the 2025 competitiveness ranking
of ten countries.

To reduce subjective bias and enhance robustness, an integrated evaluation framework combining
the Entropy Weight Method (EWM), TOPSIS, and Grey Relational Analysis (GRA) is adopted.

\subsection{Entropy-Based Weighting}

The entropy weight method is derived from information theory and assigns indicator weights
according to their dispersion across countries. Indicators with higher variability contain
more effective information for discrimination and thus receive larger weights.

Define
\begin{equation}
p_{ij}=\frac{x'_{ij}}{\sum_{i=1}^{n}x'_{ij}},\qquad 0\ln 0 := 0,
\end{equation}
where $p_{ij}$ represents the proportion of indicator $j$ contributed by country $i$.
The information entropy of indicator $j$ is computed as
\begin{equation}
e_j=-k\sum_{i=1}^{n}p_{ij}\ln p_{ij}, \quad k=\frac{1}{\ln n}.
\end{equation}
The corresponding entropy weight is given by
\begin{equation}
w_j=\frac{1-e_j}{\sum_{j=1}^{p}(1-e_j)}.
\end{equation}
If $\sum_{i=1}^{n}x'_{ij}=0$ for some indicator $j$, its weight is set to zero and the remaining
weights are renormalized.

\textit{(Result observation and interpretation to be added here.)}

\subsection{TOPSIS-Based Comprehensive Evaluation}

TOPSIS aggregates weighted indicators into a single competitiveness score by comparing each
country with an ideal reference. The method assumes that the optimal country should be closest
to the positive ideal solution and farthest from the negative ideal solution.

The weighted decision matrix is defined as
\begin{equation}
v_{ij}=w_j x'_{ij}.
\end{equation}
For each indicator $j$, the positive and negative ideal components are defined as
\begin{equation}
A_j^+=\max_{i} v_{ij},\quad A_j^-=\min_{i} v_{ij}, \qquad j=1,\dots,p,
\end{equation}
yielding the ideal vectors $A^+=(A_1^+,\dots,A_p^+)$ and $A^-=(A_1^-,\dots,A_p^-)$.

The distances from country $i$ to the ideal solutions are computed as
\begin{equation}
D_i^\pm=\sqrt{\sum_{j=1}^{p}(v_{ij}-A_j^\pm)^2},
\end{equation}
and the comprehensive competitiveness score is defined as
\begin{equation}
C_i=\frac{D_i^-}{D_i^++D_i^-}.
\end{equation}

\textit{(Result observation and interpretation to be added here.)}

\subsection{Structural Validation by Grey Relational Analysis}

Grey Relational Analysis is suitable for small-sample and partially known systems.
It evaluates the similarity of development patterns by comparing the geometric proximity
of indicator sequences.

Let the reference sequence be the ideal profile
\begin{equation}
v_{0j}=\max_{i} v_{ij}, \qquad j=1,\dots,p.
\end{equation}
Define the absolute deviation
\begin{equation}
\Delta_{ij}=|v_{0j}-v_{ij}|,
\end{equation}
and let
\begin{equation}
\Delta_{\min}=\min_{i,j}\Delta_{ij},\qquad
\Delta_{\max}=\max_{i,j}\Delta_{ij}.
\end{equation}
The grey relational coefficient is defined as
\begin{equation}
\xi_{ij}=\frac{\Delta_{\min}+\rho\Delta_{\max}}{\Delta_{ij}+\rho\Delta_{\max}}, \quad \rho=0.5,
\end{equation}
and the grey relational degree of country $i$ is calculated as
\begin{equation}
\gamma_i=\frac{1}{p}\sum_{j=1}^{p}\xi_{ij}.
\end{equation}

\textit{(Result observation and interpretation to be added here.)}

\subsection{Fusion Ranking and Reliability Framework}

To integrate distance-based performance and structural similarity, an equal-weight fusion score
is defined as
\begin{equation}
S_i=\frac{C_i+\gamma_i}{2}.
\end{equation}
The fusion score $S_i$ is used to determine the final AI competitiveness ranking for 2025.

Ranking consistency is quantified using Spearman's rank correlation coefficient
\begin{equation}
\rho_s = 1-\frac{6\sum_{i=1}^{n} d_i^2}{n(n^2-1)},
\end{equation}
where $d_i$ denotes the rank difference of country $i$ between two ranking methods.
Robustness is further assessed by perturbing a single indicator weight $w_j$ by $\pm\alpha$
(e.g., $\alpha=30\%$), followed by weight renormalization and re-ranking.

\textit{(Final ranking results and reliability analysis to be added here.)}

\subsection{Overall Summary of Task 2}

Task 2 establishes a comprehensive evaluation framework for national AI development capability
by integrating objective weighting, distance-based evaluation, and structural validation.
The framework provides a consistent and reproducible basis for generating the 2025 competitiveness
ranking, which serves as the benchmark for subsequent analysis.

\textit{(Overall result summary and interpretation to be added here.)}

% =========================
% body/7task3.tex
% =========================
\FloatBarrier
\section{Task 3: Forecasting AI Competitiveness (2026--2035)}

\begin{figure}[htbp]
  \centering
  \includegraphics[width=0.95\textwidth]{figure/task3/lct.png}
  \caption{Overall forecasting and evaluation pipeline of Task~3.}
  \label{fig:task3_framework}
\end{figure}

Fig.~\ref{fig:task3_framework} illustrates the integrated three-phase pipeline adopted in Task~3.
Phase~1 prepares historical indicator sequences and estimates GM(1,1) parameters.
Phase~2 projects country--indicator trajectories for 2026--2035.
Phase~3 evaluates annual AI competitiveness by inheriting the fixed weights and TOPSIS framework from Task~2,
yielding ranking evolution and diagnostic outputs.

Task~3 extends the 2025 evaluation to a dynamic horizon.
The key rule is consistency: \emph{the evaluation mechanism (weights and TOPSIS) is fixed},
and only the indicator trajectories evolve. Therefore, any ranking change during 2026--2035
can be attributed to data-driven indicator dynamics rather than altered standards.

\subsection{Indicator-Level Trend Prediction}

Let $x_{i,j,t}$ be indicator $j$ of country $i$ in year $t$.
For each country--indicator series over 2016--2025, we forecast $\hat{x}_{i,j,t}$ for $t=2026,\dots,2035$ independently.

Given short sequences ($T=10$), GM(1,1) is used as the primary model.
When GM(1,1) backtesting is unsatisfactory, a constrained linear trend model is used as a fallback under the same
non-negativity and truncation rules. One-step-ahead validation (train 2016--2024, predict 2025) uses MAPE as the main metric.
In our pipeline, GM(1,1) covers $44.17\%$ of the $240$ country--indicator series, while the fallback is used for $55.83\%$,
with a median MAPE of $0.1035$.

\begin{figure}[htbp]
\centering
\begin{subfigure}[t]{0.32\textwidth}
  \centering
  \includegraphics[width=\linewidth]{\detokenize{figure/task3/fig2_optimized_en_Score_Gap_Dynamics.pdf}}
  \caption{Score gaps (2026--2035).}
  \label{fig:task3_score_gap}
\end{subfigure}\hfill
\begin{subfigure}[t]{0.32\textwidth}
  \centering
  \includegraphics[width=\linewidth]{\detokenize{figure/task3/fig4_optimized_en_Slope_Chart_2025_2035.pdf}}
  \caption{2025 vs.\ 2035 slope chart.}
  \label{fig:task3_baseline_forecast}
\end{subfigure}\hfill
\begin{subfigure}[t]{0.32\textwidth}
  \centering
  \includegraphics[width=\linewidth]{\detokenize{figure/task3/fig5_optimized_en_Diagnostics_Panel.pdf}}
  \caption{Forecast diagnostics panel.}
  \label{fig:task3_diag_panel}
\end{subfigure}
\caption{Forecasting and evaluation diagnostics for Task~3.}
\label{fig:task3_diag_stack}
\end{figure}

\subsection{Annual Evaluation and Score Evolution}

After forecasting, we construct the predicted indicator matrix for each year $t$:
\[
\hat{X}_t=\bigl(\hat{x}_{i,j,t}\bigr)_{n\times p}.
\]
We keep the entropy weights from Task~2 fixed as $W=(w_1,\dots,w_p)$, and apply the same TOPSIS procedure
to obtain the annual closeness scores $C_{i,t}\in[0,1]$.

Table~\ref{tab:task3_scores_selected} reports the TOPSIS scores for representative years (2026, 2030, 2035),
while Fig.~\ref{fig:task3_score_gap} visualizes the score convergence and the evolution of cross-country gaps.

\begin{table}[htbp]
\centering
\caption{Selected TOPSIS scores $C_{i,t}$ for 2026, 2030, and 2035.}
\label{tab:task3_scores_selected}
\small
\begin{tabular}{lrrr}
\toprule
Country & 2026 & 2030 & 2035 \\
\midrule
United States & 0.653 & 0.644 & 0.633 \\
China & 0.515 & 0.505 & 0.507 \\
India & 0.213 & 0.221 & 0.244 \\
United Arab Emirates & 0.160 & 0.162 & 0.178 \\
France & 0.069 & 0.080 & 0.168 \\
Germany & 0.108 & 0.121 & 0.143 \\
United Kingdom & 0.069 & 0.073 & 0.102 \\
Canada & 0.042 & 0.057 & 0.101 \\
South Korea & 0.055 & 0.066 & 0.097 \\
Japan & 0.054 & 0.060 & 0.093 \\
\bottomrule
\end{tabular}
\end{table}

\subsection{Ranking Evolution and Stability}

Countries are ranked annually by $C_{i,t}$.
Fig.~\ref{fig:task3_rank_pair} shows both the ranking trajectories (bump chart) and the stability heatmap.
The top tier remains stable, while rank swaps occur mainly among closely competing mid-/lower-tier countries.

\begin{figure}[htbp]
\centering
\begin{subfigure}[t]{0.49\textwidth}
  \centering
  \includegraphics[width=\linewidth]{\detokenize{figure/task3/fig1_optimized_en_Rank_Evolution_Bump_Chart.pdf}}
  \caption{Bump chart (2026--2035).}
  \label{fig:task3_rank_bump}
\end{subfigure}\hfill
\begin{subfigure}[t]{0.49\textwidth}
  \centering
  \includegraphics[width=\linewidth]{\detokenize{figure/task3/fig3_optimized_en_Rank_Stability_Heatmap.pdf}}
  \caption{Rank stability heatmap.}
  \label{fig:task3_rank_heatmap}
\end{subfigure}
\caption{Ranking evolution and stability over the forecast horizon.}
\label{fig:task3_rank_pair}
\end{figure}

\begin{table}[htbp]
\centering
\caption{Rank stability summary (2026--2035).}
\label{tab:task3_rank_stats}
\small
\begin{tabular}{lrrrr}
\toprule
Country & AvgRank & StdRank & BestRank & WorstRank \\
\midrule
United States & 1.00 & 0.00 & 1 & 1 \\
China & 2.00 & 0.00 & 2 & 2 \\
India & 3.00 & 0.00 & 3 & 3 \\
United Arab Emirates & 4.00 & 0.00 & 4 & 4 \\
Germany & 5.20 & 0.40 & 5 & 6 \\
France & 5.90 & 0.54 & 5 & 7 \\
United Kingdom & 7.10 & 0.54 & 6 & 8 \\
South Korea & 8.40 & 0.49 & 8 & 9 \\
Canada & 8.90 & 1.22 & 7 & 10 \\
Japan & 9.50 & 0.50 & 9 & 10 \\
\bottomrule
\end{tabular}
\end{table}

\subsection{Interpretation and Robustness}

Because weights and evaluation rules are fixed, ranking changes come solely from predicted indicator trajectories.
Observed swaps are local (small score gaps) rather than structural reversals, consistent with the convergence pattern in Fig.~\ref{fig:task3_score_gap}.
Forecast reliability is supported by the diagnostics in Fig.~\ref{fig:task3_diag_panel}.

\subsection{Summary}

Task~3 couples indicator-level forecasting with the fixed Task~2 evaluation to project 2026--2035 competitiveness.
The results suggest stable global leadership, gradual score convergence, and limited, interpretable mid-tier rank changes,
providing the scenario baseline required by Task~4.

\FloatBarrier
\section{Task 4: Optimization of China’s AI Development Investment (2026--2035)}

In Tasks~1--3, we have established a unified analytical framework for evaluating national AI development capability:
(i) Task~1 identified a 24-indicator system and revealed key structural correlations among indicators;
(ii) Task~2 determined objective indicator weights and fixed the TOPSIS evaluation scheme;
(iii) Task~3 provided baseline and forecasted indicator trajectories for all countries up to 2035.

Under the assumption that China allocates an additional 1 trillion RMB in special funds starting from 2026, 
Task~4 aims to determine an optimal investment allocation strategy that maximizes China’s comprehensive AI competitiveness in 2035, 
while maintaining full consistency with the established evaluation framework.

\subsection{Problem Formulation}

Let $\mathbf{I} = (I_1, I_2, \ldots, I_p)^\top$ denote the investment allocation vector over the $p=24$ indicators, 
where $I_j$ represents the investment (in billion RMB) allocated to indicator $j$.
The optimization objective is defined as
\begin{equation}
\mathbf{I}^* = \arg\max_{\mathbf{I}} \; S_{\mathrm{CN}}\!\left(X^{2035}(\mathbf{I}); \mathbf{w}\right),
\end{equation}
subject to the total budget constraint
\begin{equation}
\sum_{j=1}^{p} I_j = B, \qquad B = 10000.
\end{equation}
Here, $S_{\mathrm{CN}}(\cdot)$ denotes China’s TOPSIS closeness coefficient under the fixed weight vector $\mathbf{w}$ obtained in Task~2,
and $X^{2035}(\mathbf{I})$ represents the 2035 indicator matrix in which only China’s indicator values are affected by the investment decision.

\subsection{Symbols and Data Interfaces}

To ensure comparability with previous tasks, all structural, weighting, and forecasting information is treated as exogenous input:
\begin{align}
\mathbf{w} &\leftarrow \text{Task~2 (Entropy Weight Method)},\\
\mathbf{x}^{\mathrm{base}}_{\mathrm{CN}} &\leftarrow \text{Task~3 (China baseline, 2026)},\\
X^{\mathrm{scen}}_{2035} &\leftarrow \text{Task~3 (2035 forecast scenario)},\\
\mathcal{E} &\leftarrow \text{Task~1 (strong correlation structure)}.
\end{align}

For each indicator $j$, the following parameters are introduced:
\begin{itemize}
\item $C_j$: unit investment cost required to increase indicator $j$ by one unit;
\item $\gamma_j$: time-lag discount factor reflecting realization speed;
\item $L_j$: upper bound representing saturation or feasible growth limits;
\item $I_{\min}, I_{\max}$: lower and upper bounds on single-indicator investment.
\end{itemize}

\subsection{Investment--Indicator Response Function}

Considering diminishing marginal returns and heterogeneous realization horizons across indicator types,
the incremental change of indicator $j$ induced by investment $I_j$ is modeled as
\begin{equation}
\Delta x_j(\mathbf{I}) = \frac{I_j}{C_j}
\left(1 - \frac{x^{\mathrm{base}}_j}{L_j}\right)\gamma_j,
\qquad j = 1, \ldots, p.
\end{equation}

Accordingly, China’s indicator value in 2035 after investment is given by
\begin{equation}
x^{2035}_{\mathrm{CN},j}(\mathbf{I}) 
= \min\left\{x^{\mathrm{base}}_j + \Delta x_j(\mathbf{I}), \; L_j\right\}.
\end{equation}

In vector form,
\begin{equation}
\Delta \mathbf{x}(\mathbf{I}) 
= \left(\boldsymbol{\gamma} \oslash \mathbf{C}\right)
\odot \left(\mathbf{1} - \mathbf{x}^{\mathrm{base}} \oslash \mathbf{L}\right)
\odot \mathbf{I},
\end{equation}
where $\odot$ and $\oslash$ denote element-wise multiplication and division, respectively.

\subsection{Upper Bounds and Time-Lag Settings}

Indicator upper bounds are determined according to a relative competitiveness rule:
\begin{equation}
L_j =
\begin{cases}
1.5 \, x^{\mathrm{CN}}_{j,2025}, 
& x^{\mathrm{CN}}_{j,2025} \ge x^{\mathrm{US}}_{j,2025},\\[6pt]
3.0 \, x^{\mathrm{US}}_{j,2025}, 
& x^{\mathrm{CN}}_{j,2025} < x^{\mathrm{US}}_{j,2025},
\end{cases}
\qquad
L_j \le 100 \;\; \text{(ratio-type indicators)}.
\end{equation}

Time-lag discount factors are grouped by indicator characteristics:
\begin{equation}
\gamma_j \in \{1.0,\; 0.8,\; 0.6\},
\end{equation}
corresponding respectively to short-term (e.g., infrastructure), medium-term (e.g., R\&D and applications),
and long-term (e.g., talent cultivation) effects.

\subsection{Evaluation Matrix Construction}

The 2035 evaluation matrix $X^{2035}(\mathbf{I}) \in \mathbb{R}^{n \times p}$ is constructed as follows:
all non-China rows are fixed at their forecasted values from Task~3, while China’s row is replaced by
$\mathbf{x}^{2035}_{\mathrm{CN}}(\mathbf{I})$.
This design guarantees that the comparison set and evaluation standard remain unchanged.

\subsection{TOPSIS-Based Objective Function}

Let $X = X^{2035}(\mathbf{I})$. The TOPSIS procedure follows exactly the formulation in Task~2:
\begin{equation}
\tilde{X} = X D^{-1}, 
\quad 
D = \mathrm{diag}\left(\lVert X_{:,1}\rVert_2, \ldots, \lVert X_{:,p}\rVert_2\right),
\end{equation}
\begin{equation}
V = \tilde{X} \, \mathrm{diag}(\mathbf{w}),
\quad
\mathbf{v}^+ = \max_i V_{i,:},
\quad
\mathbf{v}^- = \min_i V_{i,:},
\end{equation}
\begin{equation}
D_i^{\pm} = \lVert V_{i,:} - \mathbf{v}^{\pm} \rVert_2,
\qquad
S_i = \frac{D_i^-}{D_i^+ + D_i^-}.
\end{equation}

The optimization objective is to maximize $S_{\mathrm{CN}}$.

\subsection{Constraints}

\paragraph{Budget and Bound Constraints}
\begin{equation}
\sum_{j=1}^{p} I_j = B,
\qquad
I_{\min} \le I_j \le I_{\max}.
\end{equation}

\paragraph{Synergy Constraints}
To prevent structural imbalance caused by isolated investment surges,
synergy constraints are imposed based on strong correlations identified in Task~1:
\begin{align}
x_{\text{Large Models}} &\le 200 \, x_{\text{GPU}},\\
x_{\text{Top AI Scholars}} &\le 5.0 \, x_{\text{Researchers}},\\
x_{\text{AI Publications}} &\le 0.24 \, x_{\text{Researchers}},\\
x_{\text{AI Enterprises}} &\le 78 \, x_{\text{AI Market}},\\
x_{\text{AI Datasets}} &\le 0.75 \, x_{\text{Enterprise R\&D}}.
\end{align}

\subsection{Solution Method}

The resulting optimization problem is a nonlinear constrained programming problem.
It is solved using the Sequential Least Squares Programming (SLSQP) algorithm,
with an equal-allocation initialization $I_j = B/p$,
a maximum of 500 iterations, and a convergence tolerance of $10^{-6}$.

This formulation yields a reproducible investment allocation strategy fully consistent
with the structural insights, evaluation methodology, and forecast scenarios established in Tasks~1--3.

\section{Conclusions and Implications}

This study develops a \textbf{unified, data-driven} modeling framework for evaluating, comparing, forecasting, and optimizing national artificial intelligence (AI) development capability. Built upon a consistent system of \textbf{24 indicators} and a unified evaluation protocol, the framework integrates factor identification (Problem~1), comprehensive evaluation (Problem~2), temporal forecasting (Problem~3), and resource optimization (Problem~4) into a \textbf{closed-loop analytical pipeline}, ensuring cross-country and cross-period comparability as well as full result traceability.

\subsection*{$\blacktriangleright$ Key Findings by Problem}

\subsubsection*{$\blacklozenge$ Problem~1.}
National AI development is not driven by isolated factors but emerges from a tightly coupled system of infrastructure, human capital, policy environment, and innovation output. Multiple indicator pairs exhibit strong positive Pearson correlations (with several $|r|>0.7$), highlighting the necessity of \textbf{synergistic} investment strategies rather than fragmented interventions.

\subsubsection*{$\blacklozenge$ Problem~2.}
The 2025 evaluation reveals a clear stratification of AI competitiveness: the United States ranks first (TOPSIS score $=0.641$), China second ($0.510$), and India third ($0.210$). The high consistency between TOPSIS and Grey Relational Analysis rankings confirms the \textbf{robustness} of the results, suggesting that the observed hierarchy is primarily driven by long-term structural advantages rather than short-term fluctuations.

\subsubsection*{$\blacklozenge$ Problem~3.}
Extending the analysis to 2026--2035, the forecasting results indicate strong inertia in the global AI competitiveness landscape under \textbf{structural-stability assumptions}. Leading countries (the United States, China, and India) maintain their advantages in the medium term, while only limited rank exchanges occur among mid-tier countries. Backtesting diagnostics report a median forecasting error of $\textbf{MAPE}=0.1035$ (10.35\%), indicating acceptable predictive accuracy.

\subsubsection*{$\blacklozenge$ Problem~4.}
Under the scenario of an additional CNY~1 trillion investment for China starting in 2026, the optimization results recommend prioritizing infrastructure development (32.33\%, approximately CNY~323.3~billion), followed by talent cultivation and policy support (each 17.39\%). This allocation reflects the high marginal returns of \textbf{computational capacity} and foundational capabilities, while underscoring the complementary roles of institutional design and human capital.

\subsection*{$\blacktriangleright$ Methodological Contributions}

Methodologically, the proposed framework integrates \textbf{entropy-based objective weighting (EWM)}, multi-model cross-validation (TOPSIS and Grey Relational Analysis), time-series forecasting (GM(1,1) with backtesting diagnostics), and nonlinear constrained optimization (SLSQP). Robustness checks and error diagnostics support the credibility of the findings, indicating that the observed ranking patterns are more likely to reflect structural characteristics rather than methodological artifacts.

\subsection*{$\blacktriangleright$ Limitations}

Several limitations should be acknowledged. First, the analysis relies primarily on publicly available data, which may be subject to time lags and may not fully capture recent technological breakthroughs or abrupt policy changes. Second, the forecasting framework assumes structural stability and does not explicitly model disruptive shocks. Third, the investment-response mechanism adopts simplifying assumptions (e.g., diminishing marginal returns), whereas real-world input--output dynamics may involve more complex institutional and market feedbacks.

\subsection*{$\blacktriangleright$ Policy Implications}

Based on the findings, the following recommendations are offered:
\begin{enumerate}
    \item Formulate \textbf{long-term} AI development strategies rather than fragmented short-term interventions;
    \item Prioritize investment in \textbf{computational infrastructure}, allocating above-average shares in the short to medium term to alleviate binding constraints;
    \item Coordinate investments in talent, policy support, and R\&D so that financial, human, and institutional capacities reinforce one another;
    \item For latecomer countries, design catch-up strategies combining large-scale investment, institutional innovation, and international cooperation to mitigate path-dependence effects.
\end{enumerate}

\subsection*{$\blacktriangleright$ Overall Conclusion}

Overall, this study provides a \textbf{transparent and reproducible} framework for national AI capability assessment and optimization, offering quantitative and evidence-based support for policy formulation. Despite limitations in data availability and modeling assumptions, the framework yields meaningful insights into the structural drivers and long-term evolution of national AI development and can be readily extended in future research.


\addcontentsline{toc}{section}{References}

\begin{thebibliography}{99}

\bibitem{Hwang1981}
C.-L. Hwang and K. Yoon,
\textit{Multiple Attribute Decision Making: Methods and Applications},
Berlin: Springer, 1981.

\bibitem{Shannon1948}
C. E. Shannon,
A Mathematical Theory of Communication,
\textit{The Bell System Technical Journal},
vol. 27, no. 3, pp. 379--423, 1948;
vol. 27, no. 4, pp. 623--656, 1948.

\bibitem{Deng1982}
J. Deng,
Control Problems of Grey Systems,
\textit{Systems \& Control Letters},
vol. 1, no. 5, pp. 288--294, 1982.

\bibitem{Hotelling1933}
H. Hotelling,
Analysis of a Complex of Statistical Variables into Principal Components,
\textit{Journal of Educational Psychology},
vol. 24, no. 6, pp. 417--441, 1933;
vol. 24, no. 7, pp. 498--520, 1933.

\bibitem{Ward1963}
J. H. Ward,
Hierarchical Grouping to Optimize an Objective Function,
\textit{Journal of the American Statistical Association},
vol. 58, no. 301, pp. 236--244, 1963.

\bibitem{Nocedal2006}
J. Nocedal and S. J. Wright,
\textit{Numerical Optimization}, 2nd ed.,
New York: Springer, 2006.

\bibitem{Kraft1988}
D. Kraft,
\textit{A Software Package for Sequential Quadratic Programming},
Technical Report, 1988.

\bibitem{Virtanen2020}
P. Virtanen, R. Gommers, T. E. Oliphant, et al.,
SciPy 1.0: Fundamental Algorithms for Scientific Computing in Python,
\textit{Nature Methods},
vol. 17, pp. 261--272, 2020.

\bibitem{Harris2020}
C. R. Harris, K. J. Millman, S. J. van der Walt, et al.,
Array Programming with NumPy,
\textit{Nature},
vol. 585, pp. 357--362, 2020.

\bibitem{McKinney2010}
W. McKinney,
Data Structures for Statistical Computing in Python,
in \textit{Proceedings of the 9th Python in Science Conference (SciPy 2010)},
pp. 51--56, 2010.

\bibitem{Hunter2007}
J. D. Hunter,
Matplotlib: A 2D Graphics Environment,
\textit{Computing in Science \& Engineering},
vol. 9, no. 3, pp. 90--95, 2007.

\bibitem{HuashuCup2026}
Huashu Cup Mathematical Modeling Contest Committee,
\textit{2026 ``Huashu Cup'' International Mathematical Contest in Modeling:
Paper Format and Submission Guidelines},
January 2026.

\end{thebibliography}


\appendix
\section*{Appendix}
\addcontentsline{toc}{section}{Appendix}

\subsection*{Appendix A. Indicator System (24 Indicators) and Dimension Mapping}
\addcontentsline{toc}{subsection}{Appendix A. Indicator System (24 Indicators) and Dimension Mapping}

This paper uses a consistent set of $p=24$ benefit-type indicators across Tasks~1--4 (Table~2 in the main text).
The six-dimension TAPRIO structure is: Talent (T), Application (A), Policy (P), R\&D (R), Infrastructure (I), and Output (O).

\begin{table}[h!]
\centering
\caption{A-1. Indicator list and dimension mapping (consistent across Tasks~1--4).}
\label{tab:app_indicators}
\begin{tabular}{ll}
\toprule
Dimension & Indicator (as used in the main text) \\
\midrule
T & No. of AI Researchers \\
T & Top AI Scholars \\
T & No. of AI Graduates \\
\midrule
A & No. of AI Enterprises \\
A & AI Market Size \\
A & AI Penetration Rate \\
A & No. of LLMs \\
\midrule
P & Public Trust in AI \\
P & No. of AI Policies \\
P & AI Subsidies \\
\midrule
R & Corporate R\&D Expenditure \\
R & Government AI Investment \\
R & International AI Investment \\
\midrule
I & 5G Coverage \\
I & GPU Cluster Scale \\
I & Internet Bandwidth \\
I & Internet Penetration \\
I & Power Generation \\
I & AI Computing Platforms \\
I & No. of Data Centers \\
I & No. of TOP500 Systems \\
\midrule
O & No. of AI Books \\
O & No. of AI Datasets \\
O & GitHub Repositories \\
\bottomrule
\end{tabular}
\end{table}

\noindent
\textbf{Data scope and usage.} The evaluation set includes $n=10$ countries (Section~4.1).
Tasks~1--2 use the 2025 cross-sectional data, while Tasks~3--4 use the 2016--2025 historical panel and the 2026--2035 forecasts (Section~4.1).
All indicators are treated as benefit-type variables and normalized using min--max scaling (Section~4.3 and Eq.~(2)).

\subsection*{Appendix B. Evaluation Pipeline Reference (EWM + TOPSIS + Optional GRA)}
\addcontentsline{toc}{subsection}{Appendix B. Evaluation Pipeline Reference (EWM + TOPSIS + Optional GRA)}

For brevity, the full derivations are not duplicated here.
The paper follows a fixed evaluation standard across years and tasks:
\begin{itemize}
\item \textbf{Entropy Weight Method (EWM)}: Eqs.~(11)--(13) in Section~6.1 define the proportion matrix, entropy, and weights $w_j$.
\item \textbf{TOPSIS aggregation}: Eqs.~(14)--(17) in Section~6.1 define weighted performance, ideal solutions, distances $D_i^\pm$, and the closeness score $C_i$.
\item \textbf{Grey Relational Analysis (GRA) cross-check and fusion (optional)}: Eqs.~(18)--(20) in Section~6.1 provide the GRA degree $\gamma_i$ and fusion score $S_i=(C_i+\gamma_i)/2$. The main ranking discussion uses the TOPSIS score and reports GRA consistency as robustness evidence (Section~6.2).
\end{itemize}

\noindent
\textbf{Consistency rule used in Task~3/4.}
Task~3 keeps the weight vector $\mathbf{w}$ fixed from Task~2 and applies the same TOPSIS procedure to annual predicted matrices $\hat{X}_t$ (Section~7.2).
Task~4 also evaluates the post-investment 2035 matrix $X_{2035}(\mathbf{I})$ using the same fixed weights and TOPSIS standard (Eqs.~(30)--(32)).

\subsection*{Appendix C. Forecasting Implementation Notes (GM(1,1) + Fallback)}
\addcontentsline{toc}{subsection}{Appendix C. Forecasting Implementation Notes (GM(1,1) + Fallback)}

Task~3 forecasts each country--indicator series independently over 2016--2025 and projects 2026--2035 (Section~7.1).
Given the short time length ($T=10$), GM(1,1) is used as the primary model, with a constrained linear trend fallback when GM(1,1) backtesting is unsatisfactory.
One-step-ahead validation (train 2016--2024, predict 2025) uses MAPE as the main accuracy metric (Section~7.1 and Fig.~4c).
In the reported pipeline, GM(1,1) covers 44.17\% of series and the fallback covers 55.83\%, with a median MAPE of 0.1035.

\subsection*{Appendix D. Task 4 Optimization Settings (Condensed)}
\addcontentsline{toc}{subsection}{Appendix D. Task 4 Optimization Settings (Condensed)}

\textbf{Decision variables and budget.}
The allocation vector is $\mathbf{I}=(I_1,\dots,I_p)^\top$ over $p=24$ indicators.
Investment is measured in 100 million RMB; the total budget is $B=10000$ (Eq.~(21)).

\textbf{Objective.}
The objective maximizes China's 2035 TOPSIS closeness under the fixed weight vector $\mathbf{w}$ from Task~2:
\[
\mathbf{I}^*=\arg\max_{\mathbf{I}} S_{\mathrm{CN}}\!\left(X_{2035}(\mathbf{I});\mathbf{w}\right),
\]
as defined in Eq.~(22).

\textbf{Investment--indicator response.}
Indicator increments follow diminishing returns with saturation and time-lag discount (Eq.~(27)),
and the post-investment level is truncated by an upper bound (Eq.~(28)).
Upper bounds follow the relative-competitiveness rule in Eq.~(29), including a cap for ratio-type indicators.

\textbf{Constraints.}
Budget and per-indicator bounds are given in Eq.~(33).
Synergy constraints are ratio-type coupling constraints derived from the strong-correlation structure in Task~1 (Eq.~(34)),
used to avoid structurally imbalanced growth.

\textbf{Solver.}
The nonlinear constrained program is solved by SLSQP with equal-allocation initialization $I_j=B/p$,
maximum 500 iterations, and tolerance $10^{-6}$ (end of Section~8.1).

\subsection*{Appendix E. Reproducibility Checklist (Minimal)}
\addcontentsline{toc}{subsection}{Appendix E. Reproducibility Checklist (Minimal)}

All reported figures and tables are generated from:
\begin{itemize}
\item \textbf{Task~1 outputs:} correlation heatmap, clustering dendrogram, PCA variance and importance, and interaction network (Figs.~1--2).
\item \textbf{Task~2 outputs:} entropy weight distribution, TOPSIS/GRA ranking, and robustness statements (Fig.~3; Tables~3--4).
\item \textbf{Task~3 outputs:} forecasting diagnostics, representative-year TOPSIS scores, and rank stability summaries (Figs.~4--5; Tables~5--6).
\item \textbf{Task~4 outputs:} optimized allocation and indicator changes under the response function (Figs.~6--9; Tables~7--10).
\end{itemize}

\noindent
The evaluation standard (weights and TOPSIS rules) is fixed once established in Task~2 and is reused without modification in Tasks~3--4,
ensuring cross-year and cross-scenario comparability (Sections~7.2 and~8.1).


%%%%%%%%%%%%%%%%%%%%%%%%%%%%%%
\clearpage
\pagestyle{fancy}
\thispagestyle{fancy}

%\clearpage
\pagestyle{empty}
\thispagestyle{empty}
\section*{AI Use Report}

\noindent
In the preparation of this paper, artificial intelligence (AI) tools were used solely as
\textit{auxiliary support tools} to assist with language organization, structural refinement,
and consistency checking. The use of AI tools did \textbf{not} replace any part of the
mathematical modeling, data analysis, computational implementation, or result interpretation
conducted by the team.

\medskip
\noindent
Specifically, AI tools were used for the following purposes:
\begin{itemize}
  \item Assisting with English language polishing and improving clarity and coherence of written expressions;
  \item Providing suggestions on LaTeX formatting and structural organization of the paper;
  \item Helping to summarize and restate model descriptions and results based strictly on outputs generated by the team;
  \item Supporting logical consistency checks of explanations without introducing new assumptions or results.
\end{itemize}

\medskip
\noindent
All mathematical models, assumptions, parameter selections, data preprocessing procedures,
algorithm implementations, numerical experiments, figures, tables, and conclusions were
\textbf{independently designed, implemented, and verified by the team}. No AI tool was used
to generate raw data, perform numerical computation, determine model structure, or make
substantive analytical decisions.

\medskip
\noindent
The team takes full responsibility for the originality, correctness, and integrity of the
models and results presented in this paper. The use of AI tools strictly complies with the
competition requirements and academic integrity standards.


%\end{CJK}
\end{document}
\end